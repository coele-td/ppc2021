%%%%%%%%%%%%%%%%%%%%%%%%%%%%%%%%%%%%%%%%%%%%%%%%%%%%%%%%%%%%%%%%%%%%%%%%%%%%%%%%%%%%%%%%%%%%%%%%%%%
%   Comandos customizados
%
%   Este arquivo lista comandos customizados para utilizar no PCC.
%   Muitos comandos utilizam diretamente o arquivo Dados/unidadesCurriculares.csv
% 

%%%%%%%%%%%%%%%%%%%%%%%%%%%%%%%%%%%%%%%%%%%%%%%%%%%%%%%%%%%%%%%%%%%%%%%%%%%%%%%%%%%%%%%%%%%%%%%%%%%
% Escreve o nome por extenso da UTFPR:
\newcommand{\utf}{Universidade Tecnológica Federal do Paraná}

%%%%%%%%%%%%%%%%%%%%%%%%%%%%%%%%%%%%%%%%%%%%%%%%%%%%%%%%%%%%%%%%%%%%%%%%%%%%%%%%%%%%%%%%%%%%%%%%%%%
% Coloca um checkmark caso a disciplina seja extensionista:
\newcommand{\ifext}[2]{\ifcsvstrcmp{\ext}{0}{#1}{#2}} % Utilizado nas tabelas das unidades curriculares dos períodos

%%%%%%%%%%%%%%%%%%%%%%%%%%%%%%%%%%%%%%%%%%%%%%%%%%%%%%%%%%%%%%%%%%%%%%%%%%%%%%%%%%%%%%%%%%%%%%%%%%%
% Escreve a área da optativa
\newcommand{\optativa}[1]{
    \ifnum #1 = 21
        Controle e automação\fi
    \ifnum #1 = 22
        Trilha de Computação\fi
    \ifnum #1 = 23
        Trilha de Eletrônica\fi
    \ifnum #1 = 24
        Trilha de Eletrotécnica\fi
    \ifnum #1 = 31
        Optativas de Humanidades\fi
    \ifnum #1 = 32
        Metodologia de Ensino Inovador da UTFPR\fi
    \ifnum #1 = 33
        Não definido\fi
    \ifnum #1 = 41
        Optativas de Ciências do Ambiente\fi
}
% Este comando define a macro \optativa, que aceita um argumento numérico e retorna uma string representando a categoria de disciplina correspondente.
% Os números fornecidos como argumento representam períodos de disciplinas na coluna "período" da planilha unidadesCurriculares.csv localizada na pasta Dados.
% Quando o período é maior que 10, o valor do período indica uma categoria específica de disciplina, conforme mostrado neste comando.
% 
% Neste código, cada condição é verificada separadamente usando o comando \ifnum, seguido pelo valor correspondente da categoria de disciplina e, em seguida, fechado com \fi.
% 
% Para adicionar mais categorias, siga os passos abaixo:
% 
% 1. Adicione uma nova linha após as categorias existentes, usando o mesmo formato.
% 2. Substitua o número (#1) no \ifnum pelo valor numérico correspondente à nova categoria.
% 3. Substitua o texto entre chaves {} pelo nome da nova categoria de disciplina e adicione \fi no final da linha.
% 
% Por exemplo, para adicionar uma categoria "Exemplo" com o valor numérico 42, adicione a seguinte linha:
% \ifnum #1 = 42 Exemplo\fi
%
% Certifique-se de que os valores numéricos usados sejam únicos e não se sobreponham aos valores existentes.

%%%%%%%%%%%%%%%%%%%%%%%%%%%%%%%%%%%%%%%%%%%%%%%%%%%%%%%%%%%%%%%%%%%%%%%%%%%%%%%%%%%%%%%%%%%%%%%%%%%
% Escreve o período por extenso
\newcommand{\pernum}[1]{% \pernum{<número do período de 1 à 10>}
    \ifthenelse{#1 < 1 \OR #1 > 10}{%
        !Erro!% Mensagem de erro para valores fora do intervalo de 1 a 10
    }{%
        \ifnum #1 = 1
            Primeiro\fi
        \ifnum #1 = 2
            Segundo\fi
        \ifnum #1 = 3
            Terceiro\fi
        \ifnum #1 = 4
            Quarto\fi
        \ifnum #1 = 5
            Quinto\fi
        \ifnum #1 = 6
            Sexto\fi
        \ifnum #1 = 7
            Sétimo\fi
        \ifnum #1 = 8
            Oitavo\fi
        \ifnum #1 = 9
            Nono\fi
        \ifnum #1 = 10
            Décimo\fi
    }
}
% Este comando define a macro \pernum, que aceita um argumento numérico e retorna o período por extenso correspondente (de 1 a 10).
% Utiliza-se o pacote ifthen para verificar se o valor fornecido está fora do intervalo de 1 a 10 e exibir uma mensagem de erro, se necessário.
% Quando o valor está dentro do intervalo válido, o comando verifica cada condição separadamente usando o comando \ifnum, seguido pelo valor correspondente do período por extenso e, em seguida, fechado com \fi.
%
% Para utilizar o comando \pernum, insira \pernum{<número do período de 1 à 10>}
%
% Exemplo de uso:
% \pernum{3} retorna "Terceiro"
%
% Se o valor fornecido estiver fora do intervalo de 1 a 10, a mensagem "!Erro!" será exibida.

%%%%%%%%%%%%%%%%%%%%%%%%%%%%%%%%%%%%%%%%%%%%%%%%%%%%%%%%%%%%%%%%%%%%%%%%%%%%%%%%%%%%%%%%%%%%%%%%%%%
% Imprime a planilha do período
\newcommand{\tabelaPeriodo}[2]{% \tabelaPeriodo{<número do período de 1 à 10>}{<arquivo csv>}
    
    \begin{tabularx}{\textwidth}{>{\centering\arraybackslash}X >{\centering\arraybackslash}X c||ccc}\toprule

        % Primeira linha:
    	\multicolumn{3}{c||}{\bfseries \pernum{#1} Período}	&	\multicolumn{3}{c}{\bfseries Carga horária (h)}\\
    	\midrule

        % Variáveis da tabela em negrito:
    	\bfseries Unidade Curricular & \bfseries Área do Conhecimento	& \bfseries Extensionista & \bfseries AT$^1$ & \bfseries AP$^2$& \bfseries Total$^3$\\
    	\midrule
    
        \csvreader[ head to column names,
                    separator=pipe,
                    %before first line=\\\rowcolor{gray!10},                 
                    late after line=\csvifoddrow{\\}{\\\rowcolor{gray!10}}, 
                    filter test=\ifnumequal{\periodo}{#1}]%
                    {#2}{}%
    	{\hyperref[qua:uc\codigo]{\nome}	& \areaConhecimento	& \ifext{-}{\checkmark} & \chTeorica	& \chPratica	& \chtotal }%

    	\midrule

        \rowcolor{white}\multicolumn{3}{r}{\bfseries Totais:} & \bfseries \the\value{horasAT#1} & \bfseries \the\value{horasAP#1}  & \bfseries \the\value{horasT#1}\\
        
        % \midrule
        
        % \rowcolor{white}\multicolumn{6}{r}{\bfseries Carga horária presencial total:} & \bfseries \the\value{horasP#1}\\
        % \rowcolor{white}\multicolumn{6}{r}{\bfseries Carga horária não-presencial total:} & \bfseries \the\value{horasANP#1}\\
    	% \rowcolor{white}\multicolumn{6}{r}{\bfseries Carga horária total:} & \bfseries \the\value{horasP#1}\\

    	\bottomrule  

    	\multicolumn{6}{l}{$^1$ \tiny AT - Atividades Teóricas}\\
    	\multicolumn{6}{l}{$^2$ \tiny AP - Atividades Práticas}\\
        \multicolumn{6}{l}{$^3$\tiny Total = AT + AP}\\	
    \end{tabularx}    
}	

% Imprime a planilha de trilha ou Optativa
\newcommand{\tabelaTrilha}[2]{% \tabelaTrilha{<número da Optativa de 21 à 49>}{<arquivo csv>}
    
    \begin{tabularx}{\textwidth}{>{\centering\arraybackslash}X >{\centering\arraybackslash}X c||ccc}\toprule

        % Primeira linha:
    	\multicolumn{3}{c||}{\bfseries \optativa{#1}}	&	\multicolumn{3}{c}{\bfseries Carga horária (h)}\\
    	\midrule

        % Variáveis da tabela em negrito:
    	\bfseries Unidade Curricular & \bfseries Área do Conhecimento	& \bfseries Extensionista & \bfseries AT$^1$ & \bfseries AP$^2$& \bfseries Total$^3$\\
    	\midrule
    
        \csvreader[ head to column names,
                    separator=pipe,
                    %before first line=\\\rowcolor{gray!10},                 
                    late after line=\csvifoddrow{\\}{\\\rowcolor{gray!10}}, 
                    filter test=\ifnumequal{\periodo}{#1}]%
                    {#2}{}%
    	{\hyperref[qua:uc\codigo]{\nome}	& \areaConhecimento	& \ifext{-}{\checkmark} & \chTeorica	& \chPratica	& \chtotal }%

    	\midrule

        \rowcolor{white}\multicolumn{3}{r}{\bfseries Totais:} & \bfseries \the\value{horasAT#1} & \bfseries \the\value{horasANP#1} & \bfseries \the\value{horasT#1}\\
        
        % \midrule
        
        % \rowcolor{white}\multicolumn{6}{r}{\bfseries Carga horária presencial total:} & \bfseries \the\value{horasP#1}\\
        % \rowcolor{white}\multicolumn{6}{r}{\bfseries Carga horária não-presencial total:} & \bfseries \the\value{horasANP#1}\\
    	% \rowcolor{white}\multicolumn{6}{r}{\bfseries Carga horária total:} & \bfseries \the\value{horasP#1}\\

    	\bottomrule  

    	\multicolumn{6}{l}{$^1$ \tiny AT - Atividades Teóricas}\\
    	\multicolumn{6}{l}{$^2$ \tiny AP - Atividades Práticas}\\
        \multicolumn{6}{l}{$^3$ \tiny Total = AT + AP}\\	
    \end{tabularx}  
}  

% Imprime a planilha Optativas de humanas
\newcommand{\tabelaOptHum}[2]{% \tabelaOptHum{<número da Optativa de 21 à 49>}{<arquivo csv>}
    
    \begin{tabularx}{\textwidth}{>{\centering\arraybackslash}X >{\centering\arraybackslash}X c||ccc}\toprule

        % Primeira linha:
    	\multicolumn{3}{c||}{\bfseries \optativa{#1}}	&	\multicolumn{3}{c}{\bfseries Carga horária (h)}\\
    	\midrule

        % Variáveis da tabela em negrito:
    	\bfseries Unidade Curricular & \bfseries Área & \bfseries Extensionista & \bfseries AT$^1$ & \bfseries AP$^2$& \bfseries Total$^3$\\
    	\midrule
    
        \csvreader[ head to column names,
                    separator=pipe,
                    %before first line=\\\rowcolor{gray!10},                 
                    late after line=\csvifoddrow{\\}{\\\rowcolor{gray!10}}, 
                    filter test=\ifnumequal{\periodo}{#1}]%
                    {#2}{}%
    	{\hyperref[qua:uc\codigo]{\nome}	& \areaHumanas	& \ifext{-}{\checkmark} & \chTeorica	& \chPratica	& \chtotal }%

    	\midrule

        \rowcolor{white}\multicolumn{3}{r}{\bfseries Totais:} & \bfseries \the\value{horasAT#1} & \bfseries \the\value{horasAP#1} & \bfseries \the\value{horasT#1}\\
        
        % \midrule
        
        % \rowcolor{white}\multicolumn{6}{r}{\bfseries Carga horária presencial total:} & \bfseries \the\value{horasP#1}\\
        % \rowcolor{white}\multicolumn{6}{r}{\bfseries Carga horária não-presencial total:} & \bfseries \the\value{horasANP#1}\\
    	% \rowcolor{white}\multicolumn{6}{r}{\bfseries Carga horária total:} & \bfseries \the\value{horasP#1}\\

    	\bottomrule  

    	\multicolumn{6}{l}{$^1$ \tiny AT - Atividades Teóricas}\\
    	\multicolumn{6}{l}{$^2$ \tiny AP - Atividades Práticas}\\    	
        \multicolumn{6}{l}{$^3$ \tiny Total = AT + AP}\\
    \end{tabularx}  
}  

% Imprime a planilha do ciclo de humanidades humanas
\newcommand{\tabelaHum}[1]{% \tabelaHum{<arquivo csv>}
    
    \begin{tabularx}{\textwidth}{>{\centering\arraybackslash}X >{\centering\arraybackslash}X c||ccc}\toprule

        % Primeira linha:
    	\multicolumn{3}{c||}{\bfseries Unidade Curricular}	&	\multicolumn{3}{c}{\bfseries Carga horária (h)}\\
    	\midrule

        % Variáveis da tabela em negrito:
    	\bfseries Unidade Curricular & \bfseries Área & \bfseries Extensionista & \bfseries AT$^1$ & \bfseries AP$^2$& \bfseries Total$^3$\\
    	\midrule
    
        \csvreader[ head to column names,
                    separator=pipe,
                    %before first line=\\\rowcolor{gray!10},                 
                    late after line=\csvifoddrow{\\}{\\\rowcolor{gray!10}}, 
                    filter expr={test{\ifnumgreater{\humanidades}{0}} and test{\ifnumcomp{\periodo}{<}{12}} }]%
                    {#1}{}%
    	{\hyperref[qua:uc\codigo]{\nome}	& \areaHumanas	& \ifext{-}{\checkmark} & \chTeorica	& \chPratica	& \chtotal }%

    	\midrule

        \rowcolor{white}\multicolumn{3}{r}{\bfseries Totais:} & \bfseries \the\value{horasH:AT} & \bfseries \the\value{horasH:AP} & \bfseries \the\value{horasH:T}\\
        
        % \midrule
        
        % \rowcolor{white}\multicolumn{6}{r}{\bfseries Carga horária presencial total:} & \bfseries \the\value{horasP#1}\\
        % \rowcolor{white}\multicolumn{6}{r}{\bfseries Carga horária não-presencial total:} & \bfseries \the\value{horasANP#1}\\
    	% \rowcolor{white}\multicolumn{6}{r}{\bfseries Carga horária total:} & \bfseries \the\value{horasP#1}\\

    	\bottomrule  

    	\multicolumn{6}{l}{$^1$ \tiny AT - Atividades Teóricas}\\
    	\multicolumn{6}{l}{$^2$ \tiny AP - Atividades Práticas}\\
        \multicolumn{6}{l}{$^3$ \tiny Total = AT + AP}\\	
    \end{tabularx}  
}  


% Imprime as ementas de um período
\newcommand{\imprimeEmentas}[2]{% % \imprimeEmentas{<número do período de 1 à 10>}{<arquivo csv>}
    \csvreader[head to column names, separator=pipe, filter test=\ifnumequal{\periodo}{#1}]{#2}{}{%
	
		\begin{quadro}[!ht]
            \footnotesize
			\centering
			\caption{Dados estruturais da unidade curricular de \nome}
			\label{qua:uc\codigo}
			\begin{tabularx}{\textwidth}{|c|>{\centering}m{4cm}|c|c|}
            %\begin{tabularx}{\textwidth}{|>{\centering\arraybackslash}X|>{\centering\arraybackslash}X|>{\centering\arraybackslash}X|>{\centering\arraybackslash}X|}
				\hline
				\multicolumn{4}{|>{\centering}X|}{\cellcolor[HTML]{656565}\bfseries \color{white} \vspace{5pt} \nome \vspace{5pt}}\\ \hline\hline				
				\cellcolor[HTML]{C0C0C0}Período                 & \hyperref[tab:per\periodo]{\pernum{\periodo}}			    & \cellcolor[HTML]{C0C0C0}Código			& \codigo		    \\ \hline
				\cellcolor[HTML]{C0C0C0}Referência na matriz    & \referencia		                                        & \cellcolor[HTML]{C0C0C0}Pré-requisito		& \prereq		    \\ \hline
				\cellcolor[HTML]{C0C0C0}Área do Conhecimento    & \areaConhecimento	                                        & \cellcolor[HTML]{C0C0C0}Modalidade 		& \modalidade	    \\ \hline				
				\cellcolor[HTML]{C0C0C0}Extensionista 			& \ifext{não}{sim}                                          & \cellcolor[HTML]{C0C0C0}Idioma        	& \idioma           \\ \hline\hline
				\multirow{2}{*}{\bfseries Carga Horária}        & \multicolumn{2}{c|}{\bfseries Atividade Teórica - AT}                                                  & \chTeorica{} h    \\ \cline{2-4}
				                                                & \multicolumn{2}{c|}{\bfseries Atividade Prática - AP}                                                  & \chPratica{} h    \\ \cline{2-4} \hline
				                                                

				\multicolumn{4}{|c|}{\cellcolor[HTML]{C0C0C0}\textbf{Carga horária total: \chtotal{} h}}  																		            \\ \hline\hline
				\multicolumn{4}{|>{\centering}X|}{\bfseries \vspace{5pt} Ementa \vspace{5pt}}                                                                       			            \\ \hline
				\multicolumn{4}{|>{\hsize=\hsize}X|}{\ementa}                                                                                                                               \\ \hline
			\end{tabularx}
			\fonte{Autoria própria}
		\end{quadro}
	
	}
}

% Imprime as ementas de optativas
\newcommand{\imprimeEmentasOpt}[2]{% % \imprimeEmentasOpt{<número do período de 21 à 49>}{<arquivo csv>}
    \csvreader[head to column names, separator=pipe, filter test=\ifnumequal{\periodo}{#1}]{#2}{}{%
	
		\begin{quadro}[!ht]
            \footnotesize
			\centering
			\caption{Dados estruturais da unidade curricular optativa de \nome}
			\label{qua:uc\codigo}
			\begin{tabularx}{\textwidth}{|c|>{\centering}m{4cm}|c|c|}
            %\begin{tabularx}{\textwidth}{|>{\centering\arraybackslash}X|>{\centering\arraybackslash}X|>{\centering\arraybackslash}X|>{\centering\arraybackslash}X|}
				\hline
				\multicolumn{4}{|>{\centering}X|}{\cellcolor[HTML]{656565}\bfseries \color{white} \vspace{5pt} \nome \vspace{5pt}}\\ \hline\hline				
				\cellcolor[HTML]{C0C0C0}Tipo de Optativa        & \optativa{\periodo}			        & \cellcolor[HTML]{C0C0C0}Código			& \codigo		    \\ \hline
				\cellcolor[HTML]{C0C0C0}Referência na matriz    & N/A		                                        & \cellcolor[HTML]{C0C0C0}Pré-requisito		& \prereq		    \\ \hline
				\cellcolor[HTML]{C0C0C0}Área do Conhecimento    & \areaConhecimento	                                        & \cellcolor[HTML]{C0C0C0}Modalidade 		& \modalidade	    \\ \hline				
				\cellcolor[HTML]{C0C0C0}Extensionista 			& \ifext{não}{sim}                                          & \cellcolor[HTML]{C0C0C0}Idioma        	& \idioma           \\ \hline\hline
				\multirow{2}{*}{\bfseries Carga Horária}        & \multicolumn{2}{c|}{\bfseries Atividade Teórica - AT}                                                  & \chTeorica{} h    \\ \cline{2-4}
				                                                & \multicolumn{2}{c|}{\bfseries Atividade Prática - AP}                                                  & \chPratica{} h    \\ \cline{2-4}\hline
				                                                

				\multicolumn{4}{|c|}{\cellcolor[HTML]{C0C0C0}\textbf{Carga horária total: \chtotal{} h}}  																		            \\ \hline\hline
				\multicolumn{4}{|>{\centering}X|}{\bfseries \vspace{5pt} Ementa \vspace{5pt}}                                                                       			            \\ \hline
				\multicolumn{4}{|>{\hsize=\hsize}X|}{\ementa}                                                                                                                               \\ \hline
			\end{tabularx}
			\fonte{Autoria própria}
		\end{quadro}
	
	}
}

% Imprime as ementas de optativas de Humanidades
\newcommand{\imprimeEmentasHum}[2]{% % \imprimeEmentasHum{<número do período de 21 à 49>}{<arquivo csv>}
    \csvreader[head to column names, separator=pipe, filter test=\ifnumequal{\periodo}{#1}]{#2}{}{%
	
		\begin{quadro}[!ht]
            \footnotesize
			\centering
			\caption{Dados estruturais da unidade curricular optativa de \nome}
			\label{qua:uc\codigo}
			\begin{tabularx}{\textwidth}{|c|>{\centering}m{4cm}|c|c|}
            %\begin{tabularx}{\textwidth}{|>{\centering\arraybackslash}X|>{\centering\arraybackslash}X|>{\centering\arraybackslash}X|>{\centering\arraybackslash}X|}
				\hline
				\multicolumn{4}{|>{\centering}X|}{\cellcolor[HTML]{656565}\bfseries \color{white} \vspace{5pt} \nome \vspace{5pt}}\\ \hline\hline				
				\cellcolor[HTML]{C0C0C0}Tipo de Optativa        & \optativa{\periodo}			        & \cellcolor[HTML]{C0C0C0}Código			& \codigo		    \\ \hline
				\cellcolor[HTML]{C0C0C0}Referência na matriz    & N/A		                                        & \cellcolor[HTML]{C0C0C0}Pré-requisito		& \prereq		    \\ \hline
				\cellcolor[HTML]{C0C0C0}Área do Conhecimento    & \areaHumanas	                                            & \cellcolor[HTML]{C0C0C0}Modalidade 		& \modalidade	    \\ \hline				
				\cellcolor[HTML]{C0C0C0}Extensionista 			& \ifext{não}{sim}                                          & \cellcolor[HTML]{C0C0C0}Idioma        	& \idioma           \\ \hline\hline
				\multirow{3}{*}{\bfseries Carga Horária}        & \multicolumn{2}{c|}{\bfseries Atividade Teórica - AT}                                                  & \chTeorica{} h    \\ \cline{2-4}
				                                                & \multicolumn{2}{c|}{\bfseries Atividade Prática - AP}                                                  & \chPratica{} h    \\ \cline{2-4}\hline
				                                                

				\multicolumn{4}{|c|}{\cellcolor[HTML]{C0C0C0}\textbf{Carga horária total: \chtotal{} h}}  																		            \\ \hline\hline
				\multicolumn{4}{|>{\centering}X|}{\bfseries \vspace{5pt} Ementa \vspace{5pt}}                                                                       			            \\ \hline
				\multicolumn{4}{|>{\hsize=\hsize}X|}{\ementa}                                                                                                                               \\ \hline
			\end{tabularx}
			\fonte{Autoria própria}
		\end{quadro}
	
	}
}

%Lista as disciplinas de um período em texto corrido, referenciando o quadro das características estruturais
% \newcommand{\tiraespaco}{)}
% \newcommand{\listaUC}[2]{\!\!\!
%     \csvreader[ head to column names, 
%                 separator=pipe,                
%                 late after line={{,}\space},
%                 late after first line={{,}\space},
%                 late after last line={},
%                 filter test=\ifnumequal{\periodo}{#1}]{#2}{}{%
%         \nome\space(\autoref{qua:uc\codigo}\tiraespaco
%     }\!\!
% }

%Lista as disciplinas de um período em texto corrido, referenciando o quadro das características estruturais
\newcounter{totalrows}
\newcounter{currentrow}
\newcommand{\listaUC}[2]{%
    \setcounter{totalrows}{0}%
    \setcounter{currentrow}{0}%
    \csvreader[ head to column names,%
                separator=pipe,%
                filter test=\ifnumequal{\periodo}{#1}]{#2}{}{\stepcounter{totalrows}}%
    \csvreader[ head to column names,%
                separator=pipe,%
                late after line={\stepcounter{currentrow}\ifnumequal{\the\value{currentrow}}{\the\value{totalrows}-1}{ e\space}{,\space}},%
                late after last line={},%
                filter test=\ifnumequal{\periodo}{#1}]{#2}{}{\nome\space(\autoref{qua:uc\codigo})}%
}%




%\newcommand{\tiraespaco}{)}
% \newcommand{\listaUC}[2]{
    
%     \csvreader[ head to column names,
%                 before first line=\autoref{qua:uc\codigo},
%                 after line=\space{à} \autoref{qua:uc\codigo},
%                 separator=pipe, 
%                 filter test=\ifnumequal{\periodo}{#1}]{#2}{}{}
    
% }

%Retorna a percentagem do segundo argumento (#2) em relação ao terceiro (#3), o primeiro parâmetro (#1) é opcional e representa as casas decimais (padrão = 2)
\newcommand{\percentagem}[3][2]{\FPset\tempval{0}\FPeval{\tempval}{round(100*#2/#3,#1)}$\tempval$\%}


\newcommand{\dadoCHT}[1]{
    \csvreader[ 
        head to column names, 
        separator=pipe                
    ]{Dados/unidadesCurriculares.csv}{}{%
        \ifthenelse{\equal{\referencia}{#1}}
            {\chtotal}  % True
            {{}}%         % False
        %\chtotal #1,
    }%
}%

% Define uma macro para coordenadas padronizadas
\newcommand{\coord}[2]{(#1*1.1cm,#2*1.1cm)}

% Imprime quadro demonstrativo de unidade curricular com legenda
\newcommand{\quadroUnitCurricularLegenda}[1]{
    \begin{tikzpicture} 
        
        %Título
        \fill[black, blur shadow] \coord{0}{2} rectangle \coord{11.6}{2.3};       
        \node[white, align=center, text width=2.5*1.1cm, font=\sffamily\scriptsize] at \coord{5.8}{2.15} {\textbf{Legenda}};
        
        % Desenha amarelo, branco e sombra
        \fill[#1, blur shadow] \coord{0}{0} rectangle \coord{3}{1.8};       
        \fill[black] \coord{2.5}{1.5} rectangle \coord{3}{1.8};
        \fill[white] \coord{0}{0} rectangle \coord{2.5}{0.3};
        \fill[white] \coord{2.5}{0} rectangle \coord{3}{1.5};
        \draw[line width=1pt] \coord{0}{0} rectangle \coord{3}{1.8};
        
        % Linhas horizontais
        \draw \coord{0}{.3} -- \coord{3}{.3};
        \draw \coord{0}{.6} -- \coord{3}{.6};
        \draw \coord{2.5}{.9} -- \coord{3}{.9};
        \draw \coord{2.5}{1.2} -- \coord{3}{1.2};
        
        % Linhas verticais
        \draw \coord{2.5}{0} -- \coord{2.5}{1.8};
        
        % Nome da disciplina
        \node[align=center, text width=2.5*1.1cm, font=\sffamily\scriptsize] at \coord{0.5*2.5}{1.2} {\textbf{Nome da unidade curricular}};
        % Código da disciplina
        \node[align=center, text width=2.3cm, font=\sffamily\tiny] at \coord{0.5*2.5}{0.45} {Código};
        % Referência
        \node[align=center, text width=2.3cm, font=\sffamily\tiny] at \coord{2.75}{1.65} {\textcolor{white}{\fontsize{6pt}{6pt}\selectfont \textbf{R}}};
        % Carga horária total
        \node[align=center, text width=2.3cm, font=\sffamily\tiny] at \coord{2.75}{1.35} {\textcolor{red}{\fontsize{6pt}{6pt}\selectfont \textbf{CT}}};
        % Atividades práticas
        \node[align=center, text width=2.3cm, font=\sffamily\tiny] at \coord{2.75}{1.05} {\textcolor{blue}{\fontsize{6pt}{6pt}\selectfont AP}};
        % Atividades teóricas
        \node[align=center, text width=2.3cm, font=\sffamily\tiny] at \coord{2.75}{0.75} {\textcolor{violet}{\fontsize{6pt}{6pt}\selectfont AT}};
        % Extensão
        \node[align=center, text width=2.3cm, font=\sffamily\tiny] at \coord{2.75}{0.45} {\textcolor{olive}{\fontsize{6pt}{6pt}\selectfont EX}};
        % Núcleo de conteúdo
        \node[align=center, text width=2.3cm, font=\sffamily\tiny] at \coord{2.75}{0.15} {\textcolor{black}{\fontsize{6pt}{6pt}\selectfont \textbf{NC}}};
        % Pré-requisitos
        \node[align=center, text width=2.3cm, font=\sffamily\tiny] at \coord{0.5*2.5}{0.15} {\textcolor{black}{\fontsize{6pt}{6pt}\selectfont Pré-requisitos}};

        % Legenda
        \draw[->] \coord{3.1}{1.65} -- \coord{3.5}{1.65};
        \draw[->] \coord{3.1}{1.35} -- \coord{3.5}{1.35};
        \draw[->] \coord{3.1}{1.05} -- \coord{3.5}{1.05};
        \draw[->] \coord{3.1}{0.75} -- \coord{3.5}{0.75};
        \draw[->] \coord{3.1}{0.45} -- \coord{3.5}{0.45};
        \draw[->] \coord{3.1}{0.15} -- \coord{3.5}{0.15};

        % Referência
        \node[align=left, text width=2.5cm, font=\sffamily\tiny] at \coord{4.75}{1.65} {\textcolor{black}{\fontsize{6pt}{6pt}\selectfont \textbf{Referência na matriz}}};
        % Carga horária total
        \node[align=left, text width=2.5cm, font=\sffamily\tiny] at \coord{4.75}{1.35} {\textcolor{red}{\fontsize{6pt}{6pt}\selectfont \textbf{Carga horária total (h)}}};
        % Atividades práticas
        \node[align=left, text width=2.5cm, font=\sffamily\tiny] at \coord{4.75}{1.05} {\textcolor{blue}{\fontsize{6pt}{6pt}\selectfont Atividades práticas (h)}};
        % Atividades teóricas
        \node[align=left, text width=2.5cm, font=\sffamily\tiny] at \coord{4.75}{0.75} {\textcolor{violet}{\fontsize{6pt}{6pt}\selectfont Atividades teóricas (h)}};
        % Extensão
        \node[align=left, text width=2.5cm, font=\sffamily\tiny] at \coord{4.75}{0.45} {\textcolor{olive}{\fontsize{6pt}{6pt}\selectfont Extensão (h)}};
        % Núcleo de conteúdo
        \node[align=left, text width=2.5cm, font=\sffamily\tiny] at \coord{4.754}{0.15} {\textcolor{black}{\fontsize{6pt}{6pt}\selectfont \textbf{Núcleo de conteúdo}}};
        
        % Núcleos
        \draw[line width=1pt] \coord{6.5}{0} rectangle \coord{11.6}{1.8};
        \draw \coord{6.5}{.6}       -- \coord{11.6}{.6};
        \draw \coord{6.5}{1.2}      -- \coord{11.6}{1.2};
        \draw[->, line width=1pt] \coord{5.6}{0.15} -- \coord{6.4}{0.15};

        \node[align=left, text width=5.5cm, font=\sffamily\tiny] at \coord{9.1}{1.5} {\textcolor{black}{\fontsize{6pt}{6pt}\selectfont \textbf{(B) Básico:} base científica e tecnológica necessária para a formação do engenheiro eletrônico}};
        \node[align=left, text width=5.5cm, font=\sffamily\tiny] at \coord{9.1}{0.9} {\textcolor{black}{\fontsize{6pt}{6pt}\selectfont \textbf{(P) Profissionalizante geral:} conhecimentos gerais para a atuação profissional}};
        \node[align=left, text width=5.5cm, font=\sffamily\tiny] at \coord{9.1}{0.3} {\textcolor{black}{\fontsize{6pt}{6pt}\selectfont \textbf{(PE) Profissionalizante específico:} conhecimentos aprofundados para a atuação profissional}};
    \end{tikzpicture}
}

% Imprime quadro demonstrativo de unidade curricular com legenda
\newcommand{\quadroContabilizacao}{
    \begin{tikzpicture} 
        
        %Título
        \fill[black, blur shadow] \coord{0}{2} rectangle \coord{11.6}{2.3};       
        \node[white, align=center, text width=11.6cm, font=\sffamily\scriptsize] at \coord{5.8}{2.15} {\textbf{Contabilização da carga horária do curso (h)}};
        
        % Quadro esquerda
        \fill[white, blur shadow] \coord{0}{0} rectangle \coord{5.6}{1.8};
        \draw[line width=1pt] \coord{0}{0} rectangle \coord{5.6}{1.8};

        % Quadro direita
        \fill[white, blur shadow] \coord{6}{0} rectangle \coord{11.6}{1.8};
        \draw[line width=1pt] \coord{6}{0} rectangle \coord{11.6}{1.8};
        
        % Linhas horizontais
        \draw \coord{0}{.3} -- \coord{5.6}{.3};
        \draw \coord{0}{.6} -- \coord{5.6}{.6};
        \draw \coord{0}{.9} -- \coord{5.6}{.9};
        \draw \coord{0}{1.2} -- \coord{5.6}{1.2};
        \draw \coord{0}{1.5} -- \coord{5.6}{1.5};
        \draw \coord{6}{.3} -- \coord{11.6}{.3};
        \draw \coord{6}{.6} -- \coord{11.6}{.6};
        \draw \coord{6}{.9} -- \coord{11.6}{.9};
        \draw \coord{6}{1.2} -- \coord{11.6}{1.2};
        \draw \coord{6}{1.5} -- \coord{11.6}{1.5};
        
        % Linhas verticais
        \draw \coord{5}{0} -- \coord{5}{1.8};
        \draw \coord{11}{0} -- \coord{11}{1.8};
        
        
        % Unidades curriculares
        \node[align=left, text width=10cm, font=\sffamily\tiny] at \coord{4.7}{1.65} {\textcolor{black}{\fontsize{6pt}{6pt}\selectfont \textbf{Unidades curriculares}}};
        \node[align=left, text width=10cm, font=\sffamily\tiny] at \coord{9.6}{1.65} {\textcolor{black}{\fontsize{6pt}{6pt}\selectfont \textbf{\the\value{horasUC}}}};
        % Atividades práticas
        \node[align=left, text width=10cm, font=\sffamily\tiny] at \coord{4.7}{1.35} {\textcolor{black}{\fontsize{6pt}{6pt}\selectfont \textbf{Atividades práticas}}};
        \node[align=left, text width=10cm, font=\sffamily\tiny] at \coord{9.6}{1.35} {\textcolor{black}{\fontsize{6pt}{6pt}\selectfont \textbf{\the\value{horasAP}}}};
        % Atividades teóricas
        \node[align=left, text width=10cm, font=\sffamily\tiny] at \coord{4.7}{1.05} {\textcolor{black}{\fontsize{6pt}{6pt}\selectfont \textbf{Atividades teórica}}};
        \node[align=left, text width=10cm, font=\sffamily\tiny] at \coord{9.6}{1.05} {\textcolor{black}{\fontsize{6pt}{6pt}\selectfont \textbf{\the\value{horasAT}}}};
        % Unidade curriculares extensionistas
        \node[align=left, text width=10cm, font=\sffamily\tiny] at \coord{4.7}{0.75} {\textcolor{black}{\fontsize{6pt}{6pt}\selectfont \textbf{Unidades curriculares extensionistas}}};
        \node[align=left, text width=10cm, font=\sffamily\tiny] at \coord{9.6}{0.75} {\textcolor{black}{\fontsize{6pt}{6pt}\selectfont \textbf{\the\value{horasUCEXT}}}};
        % Componente curricular de extensão (apostilamento)
        \node[align=left, text width=10cm, font=\sffamily\tiny] at \coord{4.7}{0.45} {\textcolor{black}{\fontsize{6pt}{6pt}\selectfont \textbf{Componente curricular de extensão (apostilamento)}}};
        \node[align=left, text width=10cm, font=\sffamily\tiny] at \coord{9.6}{0.45} {\textcolor{black}{\fontsize{6pt}{6pt}\selectfont \textbf{\the\value{horasUCEXT}}}};
        % Total de extensão
        \node[align=left, text width=10cm, font=\sffamily\tiny] at \coord{4.7}{0.15} {\textcolor{black}{\fontsize{6pt}{6pt}\selectfont \textbf{Total de extensão}}};
        \node[align=left, text width=10cm, font=\sffamily\tiny] at \coord{9.6}{0.15} {\textcolor{black}{\fontsize{6pt}{6pt}\selectfont \textbf{\the\value{horasEXT}}}};
        
        % Optativas
        \node[align=left, text width=10cm, font=\sffamily\tiny] at \coord{10.7}{1.65} {\textcolor{black}{\fontsize{6pt}{6pt}\selectfont \textbf{Total de optativas}}};
        \node[align=left, text width=10cm, font=\sffamily\tiny] at \coord{15.6}{1.65} {\textcolor{black}{\fontsize{6pt}{6pt}\selectfont \textbf{\the\value{horasOPT}}}};
        % Trilha de aprofundamento
        \node[align=left, text width=10cm, font=\sffamily\tiny] at \coord{10.7}{1.35} {\textcolor{black}{\fontsize{6pt}{6pt}\selectfont \textbf{Trilha de aprofundamento}}};
        \node[align=left, text width=10cm, font=\sffamily\tiny] at \coord{15.6}{1.35} {\textcolor{black}{\fontsize{6pt}{6pt}\selectfont \textbf{180}}};
        % Ciclo de humanidades
        \node[align=left, text width=10cm, font=\sffamily\tiny] at \coord{10.7}{1.05} {\textcolor{black}{\fontsize{6pt}{6pt}\selectfont \textbf{Total do ciclo de humanidades}}};
        \node[align=left, text width=10cm, font=\sffamily\tiny] at \coord{15.6}{1.05} {\textcolor{black}{\fontsize{6pt}{6pt}\selectfont \textbf{\the\value{horasH}}}};
        % Optativas do ciclo de humanidades
        \node[align=left, text width=10cm, font=\sffamily\tiny] at \coord{10.7}{0.75} {\textcolor{black}{\fontsize{6pt}{6pt}\selectfont \textbf{Optativas do ciclo de humanidades}}};
        \node[align=left, text width=10cm, font=\sffamily\tiny] at \coord{15.5}{0.75} {\textcolor{black}{\fontsize{6pt}{6pt}\selectfont \textbf{\dadoCHT{E.3}}}};
        % Estágio
        \node[align=left, text width=10cm, font=\sffamily\tiny] at \coord{10.7}{0.45} {\textcolor{black}{\fontsize{6pt}{6pt}\selectfont \textbf{Estágio obrigatório}}};
        \node[align=left, text width=10cm, font=\sffamily\tiny] at \coord{15.6}{0.45} {\textcolor{black}{\fontsize{6pt}{6pt}\selectfont \textbf{\the\value{horasEST}}}};
        % Total do curso
        \node[align=left, text width=10cm, font=\sffamily\tiny] at \coord{10.7}{0.15} {\textcolor{black}{\fontsize{6pt}{6pt}\selectfont \textbf{Total do curso}}};
        \node[align=left, text width=10cm, font=\sffamily\tiny] at \coord{15.6}{0.15} {\textcolor{black}{\fontsize{6pt}{6pt}\selectfont \textbf{\the\value{horasT}}}};
        
    \end{tikzpicture}
}

% Comando para definir a cor com base em dados do CSV (ou condições predefinidas)
% \newcommand{\corQuadro}{grey}
\edef\corQuadro{gray}
\newcommand{\defineCorFromCSV}[1]{% #1 é o identificador da UC
    \renewcommand{\corQuadro}{yellow} % Cor padrão
    % \renewcommand{\tempRef}{#1}
    \csvreader[%
        head to column names,
        separator=pipe,
        filter strcmp={\referencia}{#1}
    ]{Dados/unidadesCurriculares.csv}{}{%
        % Supondo que 'ext', 'humanidades' e 'opt' sejam colunas do CSV correspondentes    
        \ifnumcomp{\ext}{>}{0}{%
            % \renewcommand{\corQuadro}{cyan}%
            \edef\corQuadro{cyan}%
        }{%
            \ifnumcomp{\humanidades}{>}{0}{%
                % \renewcommand{\corQuadro}{green}%
            \edef\corQuadro{green}        
            }{%
                \ifthenelse{\equal{\opt}{optativa}}}{%
                    % \renewcommand{\corQuadro}{grey}%
                    \edef\corQuadro{gray}
                }{%
                    % \renewcommand{\corQuadro}{yellow}%
                    \edef\corQuadro{yellow}
                }%
            }%
    }%
}

% Para indicar o período no quadro de disciplinas
\newcommand{\quadroTituloPeriodo}[2]{
    % Verifica a cor do quadro    
% \defineCorFromCSV{#1}
    
    % Realiza o cálculo e armazena o resultado na macro \myfontsize e \mylinespacing
    \pgfmathsetmacro{\myfontsize}{#2 * 6} % Por exemplo, 10pt com um fator de escala
    \pgfmathsetmacro{\mylinespacing}{\myfontsize * 1.2} % Espaçamento entre linhas baseado no tamanho da fonte
    \pgfmathsetmacro{\namefontsize}{#2 * 8} % Por exemplo, 10pt com um fator de escala
    \pgfmathsetmacro{\namelinespacing}{\myfontsize * 1.2} % Espaçamento entre linhas baseado no tamanho da fonte
    \begin{tikzpicture}[baseline=(current bounding box.north), every node/.style={font=\sffamily\fontsize{\myfontsize}{\mylinespacing}\selectfont}, scale=#2]%
        \fill[white, blur shadow] \coord{0}{0} rectangle \coord{3}{0.5};%
        \draw[line width=1pt] \coord{0}{0} rectangle \coord{3}{0.5};%
        \node[align=center, text width=#2*2.5cm] at \coord{1.5}{0.25} {\fontsize{\namefontsize}{\namelinespacing}\selectfont \textbf{#1\textordmasculine{} Período}};% Nome da disciplina
    \end{tikzpicture}
}

% Para indicar estágio, extensão e optativa de humanidades
\newcommand{\quadroComprido}[4]{ 
    
    \def\offsetX{3.2} % Ajuste baseado no tamanho dos quadros
    % Realiza o cálculo e armazena o resultado na macro \myfontsize e \mylinespacing
    \pgfmathsetmacro{\myfontsize}{#2 * 6} % Por exemplo, 10pt com um fator de escala
    \pgfmathsetmacro{\mylinespacing}{\myfontsize * 1.2} % Espaçamento entre linhas baseado no tamanho da fonte
    \pgfmathsetmacro{\namefontsize}{#2 * 8} % Por exemplo, 10pt com um fator de escala
    \pgfmathsetmacro{\namelinespacing}{\myfontsize * 1.2} % Espaçamento entre linhas baseado no tamanho da fonte

    \pgfmathsetmacro{\len}{#1*\offsetX}

    \begin{tikzpicture}[baseline=(current bounding box.north), every node/.style={font=\sffamily\fontsize{\myfontsize}{\mylinespacing}\selectfont}, scale=#2]%
        \fill[#4, blur shadow] \coord{0}{0} rectangle \coord{\len}{0.6};%
        \draw[line width=1pt] \coord{0}{0} rectangle \coord{\len}{0.6};%
        \node[align=center] at \coord{\len*0.5}{0.3} {\fontsize{\namefontsize}{\namelinespacing}\selectfont \textbf{#3}};% Nome da disciplina
    \end{tikzpicture}
}

% Comando para imprimir um quadro com informações de uma unidade curricular
% Parâmetros:
%   #1 - Referência da unidade curricular (usada para buscar no arquivo CSV)
%   #2 - Cor de fundo do quadro
\newcommand{\quadroUnitCurricular}[2]{
    % Verifica a cor do quadro    
% \defineCorFromCSV{#1}
    
    % Realiza o cálculo e armazena o resultado na macro \myfontsize e \mylinespacing
    \pgfmathsetmacro{\myfontsize}{#2 * 6} % Por exemplo, 10pt com um fator de escala
    \pgfmathsetmacro{\mylinespacing}{\myfontsize * 1.2} % Espaçamento entre linhas baseado no tamanho da fonte
    \pgfmathsetmacro{\namefontsize}{#2 * 8} % Por exemplo, 10pt com um fator de escala
    \pgfmathsetmacro{\namelinespacing}{\myfontsize * 1.2} % Espaçamento entre linhas baseado no tamanho da fonte
    % Lê informações da unidade curricular do arquivo CSV
    \csvreader[
        head to column names, 
        separator=pipe
    ]{Dados/unidadesCurriculares.csv}{}{%
        % Verifica se a referência da linha atual corresponde ao parâmetro #1
        \ifthenelse{\equal{\referencia}{#1}}{%            
            \begin{tikzpicture}[baseline=(current bounding box.north), every node/.style={font=\sffamily\fontsize{\myfontsize}{\mylinespacing}\selectfont}, scale=#2]%       
                
                
                \fill[\corQuadro, blur shadow] \coord{0}{0} rectangle \coord{3}{1.8};%
                                
                % Desenha áreas específicas do quadro (preto para o código da UC, branco para informações)
                \fill[black] \coord{2.5}{1.5} rectangle \coord{3}{1.8};% Área do código da UC
                \fill[white] \coord{0}{0} rectangle \coord{2.5}{0.3};% Área de texto inferior
                \fill[white] \coord{2.5}{0} rectangle \coord{3}{1.5};% Área lateral para informações adicionais
        
                % Contorno externo do quadro
                \draw[line width=1pt] \coord{0}{0} rectangle \coord{3}{1.8};%
                
                % Linhas internas para separar as seções do quadro
                \draw \coord{0}{.3} -- \coord{3}{.3};%
                \draw \coord{0}{.6} -- \coord{3}{.6};%
                \draw \coord{2.5}{.9} -- \coord{3}{.9};%
                \draw \coord{2.5}{1.2} -- \coord{3}{1.2};%
                \draw \coord{2.5}{0} -- \coord{2.5}{1.8};%
                % Insere informações da UC no quadro
                \node[align=center, text width=#2*2.5cm] at \coord{0.5*2.5}{1.2} {\fontsize{\namefontsize}{\namelinespacing}\selectfont \textbf{\nome}};% Nome da disciplina
                \node[align=center, text width=2.3cm] at \coord{0.5*2.5}{0.45} {\codigo};% Código da disciplina
                                
                \node[align=center, text width=2.3cm] at \coord{2.75}{1.65} {\textcolor{white}{\textbf{\referencia}}};% Referência                
                \node[align=center, text width=2.3cm] at \coord{2.75}{1.35} {\textcolor{red}{\textbf{\chtotal}}};% Carga horária total                
                \node[align=center, text width=2.3cm] at \coord{2.75}{1.05} {\textcolor{blue}{\chPratica}};% Atividades práticas                
                \node[align=center, text width=2.3cm] at \coord{2.75}{0.75} {\textcolor{violet}{\chTeorica}};% Atividades teóricas                
                \node[align=center, text width=2.3cm] at \coord{2.75}{0.45} {\textcolor{olive}{\ext}};% Extensão                
                \node[align=center, text width=2.3cm] at \coord{2.75}{0.15} {\textcolor{black}{\textbf{\tipo}}};% Núcleo de conteúdo                
                \node[align=center, text width=2.3cm] at \coord{0.5*2.5}{0.15} {\textcolor{black}{\prereq}};% Pré-requisitos
                
            \end{tikzpicture} 
        }%
    }%
           
}

\newcommand{\quadroResumoPeriodo}[2]{
    % Verifica a cor do quadro    
% \defineCorFromCSV{#1}
    
    % Realiza o cálculo e armazena o resultado na macro \myfontsize e \mylinespacing
    \pgfmathsetmacro{\myfontsize}{#2 * 6} % Por exemplo, 10pt com um fator de escala
    \pgfmathsetmacro{\mylinespacing}{\myfontsize * 1.2} % Espaçamento entre linhas baseado no tamanho da fonte
    \pgfmathsetmacro{\namefontsize}{#2 * 8} % Por exemplo, 10pt com um fator de escala
    \pgfmathsetmacro{\namelinespacing}{\myfontsize * 1.2} % Espaçamento entre linhas baseado no tamanho da fonte
    % Lê informações da unidade curricular do arquivo CSV
    
    \begin{tikzpicture}[baseline=(current bounding box.north), every node/.style={font=\sffamily\fontsize{\myfontsize}{\mylinespacing}\selectfont}, scale=#2]%       
                
        % Quadro branco com sombra
        \fill[white, blur shadow] \coord{0}{0} rectangle \coord{3}{1.5};%
                        
        % Título
        \fill[black] \coord{0}{1.2} rectangle \coord{3}{1.5};% Área do código da UC
        \node[align=center, text width=2.3cm] at \coord{1.5}{1.35} {\textcolor{white}{\textbf{Totais do #1\textordmasculine{} Período}}};% Período

        % Contorno externo do quadro
        \draw[line width=1pt] \coord{0}{0} rectangle \coord{3}{1.5};%
        
        % Linhas internas para separar as seções do quadro
        \draw \coord{0}{.3} -- \coord{3}{.3};%
        \draw \coord{0}{.6} -- \coord{3}{.6};%
        \draw \coord{0}{.9} -- \coord{3}{.9};%
        \draw \coord{0}{1.2} -- \coord{3}{1.2};%
        \draw \coord{2.5}{0} -- \coord{2.5}{1.2};%
        
                               
        \node[align=left, text width=2.7cm] at \coord{1.3}{1.05} {\textbf{C. horária total (h)}};% Carga horária total                
        \node[align=left, text width=2.7cm] at \coord{1.3}{0.75} {\textbf{C. horária prática (h)}};% Atividades práticas                
        \node[align=left, text width=2.7cm] at \coord{1.3}{0.45} {\textbf{C. horária teórica (h)}};% Atividades teóricas                
        \node[align=left, text width=2.7cm] at \coord{1.3}{0.15} {\textbf{C. horária de extensão (h)}};% Extensão
        
        \node[align=left, text width=0.5cm] at \coord{2.76}{1.05} {\textbf{\the\value{horasT#1}}};% Carga horária total                
        \node[align=left, text width=0.5cm] at \coord{2.76}{0.75} {\textbf{\the\value{horasAP#1}}};% Atividades práticas                
        \node[align=left, text width=0.5cm] at \coord{2.76}{0.45} {\textbf{\the\value{horasAT#1}}};% Atividades teóricas                
        \node[align=left, text width=0.5cm] at \coord{2.76}{0.15} {\textbf{\the\value{horasEXT#1}}};% Extensão  
        
        
    \end{tikzpicture}
           
}

% Define comando para encontrar o maior número de disciplinas em um período
\edef\maxLines{1}
\newcommand{\maxdisciplinas}{
    \newcounter{maxnum}    % Define um contador para armazenar o número máximo de disciplinas
    \setcounter{maxnum}{0} % Inicializa o contador máximo com zero
    \newcounter{tempcount} % Define um contador temporário para contar disciplinas em cada período
    \newcounter{n}         % Valor do período    
    
    % Loop para verificar cada período de 1 a 10
    \setcounter{n}{0}
    \whileboolexpr{test {\ifnumcomp{\value{n}}{<}{10}}}{% Analisa até 10º período
        \stepcounter{n}
        \setcounter{tempcount}{0} % Reseta o contador temporário para o período atual
        % Processa o arquivo CSV linha por linha
        \csvreader[
            head to column names,
            separator=pipe,
            filter expr={
                test{\ifnumequal{\periodo}{\value{n}}}
            }
        ]{Dados/unidadesCurriculares.csv}{}{
            \stepcounter{tempcount} % Incrementa o contador temporário se a condição for verdadeira
        }%
        % Após contar todas as disciplinas para um período, verifica se é o maior número
        \ifnum\value{tempcount}>\value{maxnum}
            \setcounter{maxnum}{\value{tempcount}} % Atualiza o contador máximo se o número atual é maior
        \fi
    }%
    % \the\value{maxnum} % Retorna o valor do número máximo de disciplinas
    \edef\maxLines{\arabic{maxnum}}
}


% Imprime os quadros das unidades curriculares
\newcommand{\distribuiQuadrosUC}{   
    \newcounter{periodoCtn}
    \newcounter{ucCtn}
    \setcounter{periodoCtn}{1}
    \setcounter{ucCtn}{1}
    \maxdisciplinas
    \begin{tikzpicture}[baseline=(current bounding box.north), every node/.style={font=\sffamily\fontsize{7}{7.2}\selectfont}]%
        % Definir as coordenadas iniciais
        \def\startX{0}
        \def\startY{0}
        \def\offsetX{3.6} % Ajuste baseado no tamanho dos quadros
        \def\offsetY{-2.2} % Ajuste baseado no tamanho dos quadros
        
        % Imprime a legenda
        \pgfmathsetmacro{\posX}{-2}
        \pgfmathsetmacro{\posY}{2.2}
        \fill[yellow, blur shadow] (\posX,\posY) rectangle (\posX+0.5,\posY+0.5);%
        \draw[line width=1pt] (\posX,\posY) rectangle (\posX+0.5,\posY+0.5);%
        \node[align=left, text width=2.5cm] at (\posX+2,\posY+0.25) {\textbf{Unidade curricular convencional}};

        \pgfmathsetmacro{\posX}{-2+\offsetX}
        \fill[green, blur shadow] (\posX,\posY) rectangle (\posX+0.5,\posY+0.5);%
        \draw[line width=1pt] (\posX,\posY) rectangle (\posX+0.5,\posY+0.5);%
        \node[align=left, text width=2.5cm] at (\posX+2,\posY+0.25) {\textbf{Unidade curricular de humanidades}};

        \pgfmathsetmacro{\posX}{-2+2*\offsetX}
        \fill[cyan, blur shadow] (\posX,\posY) rectangle (\posX+0.5,\posY+0.5);%
        \draw[line width=1pt] (\posX,\posY) rectangle (\posX+0.5,\posY+0.5);%
        \node[align=left, text width=2.5cm] at (\posX+2,\posY+0.25) {\textbf{Unidade curricular extensionista}};

        \pgfmathsetmacro{\posX}{-2+3*\offsetX}
        \fill[gray, blur shadow] (\posX,\posY) rectangle (\posX+0.5,\posY+0.5);%
        \draw[line width=1pt] (\posX,\posY) rectangle (\posX+0.5,\posY+0.5);%
        \node[align=left, text width=2.5cm] at (\posX+2,\posY+0.25) {\textbf{Unidade curricular optativa}};

        \draw[line width=1pt] (\posX-3*\offsetX ,\posY-0.2) -- (\posX+3.3,\posY-0.2);%

        % Imprime os títulos dos períodos
        \foreach \periodo in {1,...,10}{
            \pgfmathsetmacro{\posX}{-0.5 + (\periodo - 1) * \offsetX}
            \pgfmathsetmacro{\posY}{1.5}   
            \node[align=left] at (\posX,\posY) {\quadroTituloPeriodo{\arabic{periodoCtn}}{1}};
            % \node[circle, draw, fill=black, minimum size=2pt, inner sep=0pt] at (\posX,\posY) {};
            \stepcounter{periodoCtn}
        }
        \setcounter{periodoCtn}{1}

        % Loop através dos períodos (1 a 10) para apresentar as caixas de unidades curriculares
        \foreach \periodo in {1,...,10}{
            % Calcular a posição X baseada no período (pode precisar ajustar a lógica)
            \pgfmathsetmacro{\posX}{\startX + (\periodo - 1) * \offsetX}
            \setcounter{ucCtn}{1}
            % Aqui você precisa saber quantas UCs tem por período. Exemplo simples:
            \foreach \uc in {1,...,\maxLines}{ % Assumindo um máximo fixo de 5 UCs por período para o exemplo
                \pgfmathsetmacro{\posY}{\startY + (\uc - 1) * \offsetY}
                % Aqui você chamaria \quadroUnitCurricular{\periodo.\uc} para cada UC
                \edef\refTemp{\arabic{periodoCtn}.\arabic{ucCtn}}  
                \defineCorFromCSV{\refTemp}             
                \node[align=left] at (\posX,\posY) {\quadroUnitCurricular{\refTemp}{1}};
                % \node[circle, draw, fill=black, minimum size=2pt, inner sep=0pt] at (\posX,\posY) {};
                \stepcounter{ucCtn}       
            }
            \stepcounter{periodoCtn}            
        }
        
        % Estágio
        \pgfmathsetmacro{\posX}{\startX -0.6 + 8 * \offsetX}
        \pgfmathsetmacro{\posY}{\startY + (8 - 1.2) * \offsetY} 
        \node[align=left] at (\posX,\posY) {\quadroComprido{3}{1}{Estágio Curricular Obrigatório: \the\value{horasEST} h}{yellow}};
        
        % Optativa de humanidades
        \pgfmathsetmacro{\posX}{\startX -0.6 + 6 * \offsetX}
        \pgfmathsetmacro{\posY}{\startY + (8 - 0.8) * \offsetY} 
        \node[align=left] at (\posX,\posY) {\quadroComprido{7.1}{1}{Opativas do ciclo de humanidades: \dadoCHT{E.3} h}{gray}};

        % Extensão - apostilamento
        \pgfmathsetmacro{\posX}{\startX -0.6 + 5 * \offsetX}
        \pgfmathsetmacro{\posY}{\startY + (8 - 0.4) * \offsetY} 
        \node[align=left] at (\posX,\posY) {\quadroComprido{9.15}{1}{Atividades de extensão - apostilamento: \dadoCHT{E.2} h}{cyan}}; 

        \pgfmathsetmacro{\posX}{\startX -2.05}        
        \draw[line width=1pt] (\posX,\posY-0.6) -- (\posX+10*\offsetX-0.2,\posY-0.6);%

        % Resumos dos períodos
        \pgfmathsetmacro{\posY}{\startY + (8 + 0.4) * \offsetY}
        \setcounter{periodoCtn}{1}
        % Imprime os títulos dos períodos
        \foreach \periodo in {1,...,10}{
            \pgfmathsetmacro{\posX}{-0.5 + (\periodo - 1) * \offsetX}
            \node[align=left] at (\posX,\posY) {\quadroResumoPeriodo{\arabic{periodoCtn}}{1}};
            % \node[circle, draw, fill=black, minimum size=2pt, inner sep=0pt] at (\posX,\posY) {};
            \stepcounter{periodoCtn}
        }

        % Legenda dos quadros
        \pgfmathsetmacro{\posX}{\startX + 0.69 + (1) * \offsetX}
        \pgfmathsetmacro{\posY}{\startY + (8 + 1.5) * \offsetY}
        \node at (\posX,\posY) {\quadroUnitCurricularLegenda{yellow}};

        % Contabilização
        \pgfmathsetmacro{\posX}{\startX - 0.35 + (9) * \offsetX}
        \pgfmathsetmacro{\posY}{\startY + (8 + 1.5) * \offsetY}
        \node at (\posX,\posY) {\quadroContabilizacao};

        % logo
        \pgfmathsetmacro{\posX}{\startX + 1.4 + (3) * \offsetX}
        \pgfmathsetmacro{\posY}{\startY + (8 + 1.5) * \offsetY}
        \node at (\posX,\posY) {\includegraphics[scale=1.5]{logo2020_small.png}};

        % logo utfpr
        \pgfmathsetmacro{\posX}{\startX + 1.0 + (4.5) * \offsetX}
        \pgfmathsetmacro{\posY}{\startY + (8 + 1.5) * \offsetY}
        \node at (\posX,\posY) {\includegraphics[scale=0.18]{logo_utfpr.png}};


    \end{tikzpicture}
}

