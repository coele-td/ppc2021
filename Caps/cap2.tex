\chapter{VALORES E PRINCÍPIOS INSTITUCIONAIS}

Conforme definido em seu Plano de Desenvolvimento Institucional (PDI), para o quadriênio 2018-2022 \cite{pdiutfpr}, a UTFPR apresenta como valores e princípios institucionais a sua missão, a sua visão e seus valores fundamentais descritos a seguir:

\nomenclature[A]{PDI}{Plano de Desenvolvimento Institucional}

\begin{itemize}[label={}]
	\item \textbf{Missão}: Desenvolver a educação tecnológica de excelência por meio do ensino, pesquisa e extensão, interagindo de forma ética, sustentável, produtiva e inovadora com a comunidade para o avanço do conhecimento e da sociedade;
	
	\item \textbf{Visão}: Ser modelo educacional de desenvolvimento social e referência na área tecnológica;
	
	\item \textbf{Valores fundamentais}:
	
	\begin{enumerate}
		\item Ética: gerar e manter a credibilidade junto à sociedade;
		
		\item Desenvolvimento Humano: formar o cidadão integrado no contexto social;
		\item Integração Social: realizar ações interativas com a sociedade para o desenvolvimento social e tecnológico;
		\item Inovação: efetuar a mudança por meio da postura empreendedor;
		\item Qualidade e Excelência: promover a melhoria contínua dos serviços oferecidos para a satisfação da sociedade;
		\item Sustentabilidade: assegurar que todas as ações se observem sustentáveis nas dimensões sociais, ambientais e econômicas.
		
	\end{enumerate}

\end{itemize}

\section{VALORES/PRINCÍPIOS ORIENTADORES DA GRADUAÇÃO}

A partir da sua missão e visão, a UTFPR estabeleceu a ética, o desenvolvimento humano, a integração social, a inovação, a qualidade e excelência e a sustentabilidade, como os valores fundamentais para a constituição dos princípios e da identidade das graduações.

Os cursos de graduação da UTFPR oferecem formação de recursos humanos para os diversos setores da sociedade, notadamente, os setores da economia envolvidos com práticas tecnológicas e os setores educacionais, a partir da vivência dos estudantes com os problemas reais da sociedade, em especial, àqueles relacionados ao desenvolvimento socioeconômico local e regional, às competências de padrão internacional, ao desenvolvimento e aplicação da tecnologia, e à busca de alternativas inovadoras para a resolução de problemas técnicos e sociais (\href{https://sei.utfpr.edu.br/sei/publicacoes/controlador_publicacoes.php?acao=publicacao_visualizar&id_documento=888276&id_orgao_publicacao=0}{Resolução COGEP 90/2018, art. 1\textordmasculine}) \nocite{cogep90}.

Para a UTFPR, a formação de seus egressos passa pela sua capacidade de oferecer currículos flexíveis, de articular-se com a sociedade, de estimular a mobilidade acadêmica, de formar para sustentabilidade e interculturalidade, de provocar-se para a inovação curricular e metodológica e de uma forte busca pela internacionalização (\href{https://cloud.utfpr.edu.br/index.php/s/15P0OcMLMdt9Rv7}{PDI 2018-2022}, item 3.4\nocite{pdiutfpr}). A inserção efetiva desses princípios orientadores na dinâmica interna dos cursos de graduação, de torná-los efetivos em sala de aula, nos estudos, na produção científica, no planejamento, na formação continuada, ou seja, em todos os espaços em que atua, é responsabilidade de todos seus atores, e como isso se dará se consolida ao longo desse Projeto Pedagógico de Curso (PPC).

\nomenclature[A]{PPC}{Projeto Pedagógico de Curso}

Fica evidente nas seções 3 a 3.4 do \href{https://cloud.utfpr.edu.br/index.php/s/15P0OcMLMdt9Rv7}{PDI}, como as políticas de ensino serão operacionalizadas a partir da articulação dos valores e princípios institucionais com a formação em inovação, qualidade e excelência; para ética e sustentabilidade; em desenvolvimento humano e em integração social.

\subsection{Valores UTFPR: inovação e qualidade e excelência}

A formação em inovação, qualidade e excelência reportam à busca por mudanças envolvendo postura empreendedora e pela melhoria contínua dos serviços oferecidos para a satisfação da sociedade, conforme \href{https://cloud.utfpr.edu.br/index.php/s/15P0OcMLMdt9Rv7}{PDI 2018-2022}.

Nesse sentido, as atividades de formação envolvem permanentemente a inovação: a curricular e metodológica, no processo didático-pedagógico; no entendimento da tecnologia enquanto conjunto de conhecimentos que conduzem à inovação e contribuem para o desenvolvimento científico, econômico e social; promovendo discussões acerca do papel de cada um na construção de uma forte política de inovação na universidade.

Com intensa interação junto à inovação, a área de empreendedorismo é amplamente difundida e desenvolvida na graduação por meio de mecanismos de suporte para despertar nos estudantes, egressos e servidores da UTFPR o interesse pela área. Os mecanismos institucionais de apoio compreendem a implantação em cada um dos campi da universidade: hotel tecnológico, que viabiliza a pré-incubação para desenvolvimento de projetos e ideias com ênfase em tecnologia e inovação; incubadora de inovações tecnológicas, que viabiliza a incubação de empresas de base tecnológica da comunidade interna ou externa; Empresas Júniores, constituídas por acadêmicos das áreas de formação da UTFPR; e o Programa de Empreendedorismo e Inovação (PROEM).  De forma mais ampla, através do PROEM, professores, pesquisadores, estudantes e ex-alunos empreendedores da Instituição são motivados a desenvolver suas boas ideias a partir da estrutura e do ambiente privilegiados o surgimento de negócios e empresas no âmbito da própria Universidade.  Isto permite ao discente adquirir uma visão mercadológica e da sociedade suficiente para a tomada de atitudes empreendedoras, a capacidade de identificar e gerenciar riscos, capacidade de tomadas de decisão, capacidade de negociar, entre outras habilidades que complementam a formação do Engenheiro Empreendedor.

\nomenclature[A]{PROEM}{Programa de Empreendedorismo e Inovação}

Quanto às inovações curriculares, o curso acompanha as transformações do mundo moderno fomentando novas formas de organização do seu currículo, com abertura ao permanente processo de reexame visando a flexibilização, a compatibilização de conteúdo, a inovação, a sustentabilidade, a interdisciplinaridade e o empreendedorismo.  As inovações curriculares propostas possibilitam o ensino presencial, semipresencial e não presencial aos cursos de graduação (Resolução COGEP n\textordmasculine 90/2019 \cite{cogep90}), incluem o conceito de complemento da carga horária (CCH) que compreendem atividades desenvolvidas fora da sala de aula (\pdfmarkupcomment{Resolução COGEP no 053/2013}{não encontrei essa resolução}), permitem projetos interdisciplinares que tendem proporcionar a visão do todo e a motivação dos discentes em função de aplicações mais significativas dos conhecimentos adquiridos, permitem avanços tecnológicos pelo emprego de tecnologias de informação e comunicação (TIC) e, por fim, pela criação de diretrizes específicas para cada tipo de curso de graduação permitem a consolidação da identidade e da organicidade no contexto de uma universidade multi campi.

\nomenclature[A]{CCH}{Complemento da carga horária}
\nomenclature[A]{TIC}{Tecnologias de informação e comunicação}

No âmbito do curso há disciplinas vinculadas à inovação, qualidade e excelência integrando a matriz curricular, destacando-se as disciplinas de Economia, Gestão de Projetos e Empreendedorismo. No transcorrer do curso os acadêmicos podem manter contato com a área de inovação por meio de projetos desenvolvidos na disciplina optativa \textit{Enginnering Design Process} em conjunto com empresas com a Metodologia de Ensino Inovador da UTFPR (MEI-U), fundamentando o desenvolvimento inovador do discente. O curso preza pelo reconhecimento das melhores práticas universitárias, pautadas em aliar teoria em prática na formação do Engenheiro.  O curso possibilita o desenvolvimento de habilidade na área eletrotécnica ao criar uma trilha de disciplinas específicas para este fim atendendo a uma demanda de mão de obra local e regional. Dispondo de disciplinas com uma carga horária adequada destinada a práticas e ao desenvolvimento de projetos tendo a disposição laboratórios de ensino modernos e instrumentos. A sinergia com o mercado de trabalho está presente com ações de extensão, estágio em empresas que contribuem para a atualização tecnológica consolidando no âmbito regional como um curso público de excelência.

\nomenclature[A]{MEI-U}{Metodologia de Ensino Inovador da UTFPR}

\subsection{Valores UTFPR: ética e a sustentabilidade}

A formação para a ética está vinculada à formação integral do cidadão, desenvolve o sujeito comprometido seja no seu comportamento, na interação com o outro, ou na geração e manutenção da credibilidade junto à sociedade conforme PPI de 2018 \cite{ppiutfpr}.  A UTFPR é orientada pela ética e pela qualidade de vida de seus servidores e estudantes, prevalecendo um ambiente que visa: ao fortalecimento das relações com todos os envolvidos no desenvolvimento das atividades; à consolidação da imagem institucional e suas ações; e à melhoria contínua nos resultados institucionais.

O fortalecimento do trabalho cooperativo entre as diversas instâncias institucionais, em torno de objetivos comuns, é um direcionamento historicamente construído. O resultado do trabalho em rede permite compartilhar objetivos e procedimentos para a construção de vínculos de interdependência e de complementaridade, possibilitando que as ações realizadas e os resultados obtidos possam ir além dos limites de cada campus, alcançando e fortalecendo a Instituição em prol de suas comunidades.

Cada vez mais o engenheiro deve possuir consciência dos códigos de prática e ética que regem a sua profissão. Sendo assim, os projetos e soluções apresentadas pelos alunos do curso devem considerar esses dois aspectos como elementos norteadores das suas decisões. Especificamente, a Ética Profissional exige que se pense o ensino da Engenharia dentro de um quadro social, ultrapassando os métodos tradicionais de ensino, contextualizando o trabalho nos reflexos que a ciência e a tecnologia causam na sociedade e, mais diretamente, no papel dos que são diretamente responsáveis por introduzi-los no cotidiano de nossas vidas.  Destaca-se a importância de desenvolver nos futuros egressos, herdeiros dessa capacidade de criar a tecnologia que muda o mundo, a possibilidade de viver com um sentido maior de responsabilidade, como nos aponta \citeauthoronline{boff2017} (2017): \pdfmarkupcomment{``nessa urgência de uma visão de valores básicos para proporcionar um fundamento ético à emergente comunidade mundial''}{Verificar se a referência está certa}.

Associada à ética, a sustentabilidade é assegurada nas ações envolvendo as dimensões sociais, ambientais e econômicas. Como importante princípio, o entendimento de sustentabilidade envolve a manutenção do capital natural em sua capacidade de regeneração, reprodução e coevolução, coadunado ao conceito ampliado e integrador de \citeauthoronline{boff2017} (2017), para quem o termo sustentabilidade diz respeito a toda ação destinada a manter as condições energéticas, informacionais, físico-químicas que sustentam todos os seres, e tais condições devem servir de critério para avaliar o quanto temos progredido ou não rumo à sustentabilidade e devem igualmente servir de inspiração para realizar a sustentabilidade nos vários campos da atividade humana.

Comprometimento com as questões sociais e ambientais é esperado que o engenheiro seja capaz de avaliar os impactos sociais e ambientais provocados pelo desenvolvimento tecnológico e identificar oportunidades de atuação para o benefício da sociedade e do meio ambiente. O termo Engenharia da Sustentabilidade enfatiza a engenharia como profissão condutora da inovação tecnológica e capaz de conduzir transformações para uma sociedade sustentável. O Engenheiro deve ser capaz de atender às necessidades do presente sem comprometer a possibilidade das novas gerações atenderem às suas próprias necessidades. No caso da formação do Engenheiro Sustentável, pode-se inferir que esse engenheiro necessita possuir conhecimentos científicos, técnicos, de gestão, éticos, legais, culturais e que saiba produzir soluções que integrem esses conhecimentos e que beneficiem o ``bem comum''. 

Por ser um conhecimento transversal a qualquer curso de engenharia, os professores necessitam buscar integrar questões de sustentabilidade em suas disciplinas por meio de estudos de casos e exercícios que estimulem a reflexão do aluno. As questões ambientais necessitam estar inseridas no contexto local.  Apesar do tema da sustentabilidade estar inserido na disciplina Introdução à Engenharia Elétrica e também constar da unidade curricular? de Ciências do Ambiente. Durante a execução do curso, os professores serão instigados a propiciar aos alunos uma visão do Brasil e do mundo bem como os seus problemas e como o profissional de engenharia pode utilizar a sua criatividade e competência técnica para solucionar ou minimizar esses problemas.

\subsection{Valores UTFPR: desenvolvimento humano}

A formação em desenvolvimento humano, segundo o PDI 2018-2022 \cite{pdiutfpr} e o PPI 2018 \cite{ppiutfpr}, envolve a formação do cidadão integrado ao contexto social a partir de melhorias no processo de ensino e aprendizagem, de ações culturais, artísticas, esportivas e de todas as demais que contribuem para a permanência do estudante, para a sua qualidade de vida, o seu bem-estar individual e social e sua formação humana.  Neste sentido, a instituição prove programas: de acesso e permanência dos alunos por meio do Núcleo de Acompanhamento Psicopedagógico e Assistência Estudantil de Toledo (NUAPE-TD); de promoção da igualdade de oportunidades por meio de editais de ampla concorrência; da ampliação do atendimento presencial e a distância, online e offline por meio de TIC, mantendo a qualidade formal ou técnica; da integração entre concepção e execução, entre o pensar e o fazer, entre teoria e o contexto social pela participação em programas e projetos de ensino, de pesquisa, de desenvolvimento, de inovação e de extensão; e o desenvolvimento da consciência crítica da realidade com a  participação em atividades extracurriculares apoiados pela instituição.  Desse modo, não se deve considerar a formação humana e integral apenas como requisito para formar um bom trabalhador, um bom profissional ou um bom empreendedor. A formação integral do cidadão almejada pela UTFPR, envolve o desenvolvimento de um sujeito: autônomo, numa concepção ampliada de cidadania; preocupado com a preservação do ambiente, dos recursos naturais e das formas de vida do planeta; comprometido com ética e com qualidade de vida. Com um mercado competitivo, o engenheiro deve ser capaz de lidar com o estresse, rejeição ou falhas, suportar pressão e resolver conflitos. O engenheiro deve ser capaz também de planejar uma carreira de tal forma a atender aos seus anseios, sonhos profissionais e objetivos pessoais. Espera-se assim, que esse engenheiro encontre satisfação e realização profissional.

\nomenclature[A]{NUAPE-TD}{Núcleo de Acompanhamento Psicopedagógico e Assistência Estudantil de Toledo}

Sobre o desenvolvimento humano em uma região, as teorias de desenvolvimento econômico convencionais colocam como um dos fatores responsáveis pela falta de dinamismo de uma região, a inexistência de recursos humanos devidamente treinados e preparados, com capacidade de geração de novas tecnologias.  Não obstante, uma região possui aspectos dinâmicos e estratégicos de desenvolvimento, estabelecidos por meio da sucessão de mecanismos que induz e mobiliza o crescimento econômico regional. Além dos aspectos físicos e geográficos, as condições locais para gerar conhecimento, inovação ou fortalecer a atração de investimentos deve ser levada em consideração. O município de Toledo é considerado um polo econômico do oeste paranaense onde verifica-se uma evolução econômica e social bastante significativa no decorrer das décadas a partir do início do século XXI. Considerados os aspectos da evolução econômica, o processo de colonização iniciou-se em 1952 tendo foco na extração madeireira e a policultura para auto consumo que evoluiu a partir da década de 70 para um novo modo de produção e de organização pelas cooperativas agropecuárias e a agroindustrialização.  Em 2020, a produção agropecuária de insumos, a indústria de alimentos e a farmacêutica dinamizam a economia local, em que os produtos e serviços são exportados para outras regiões e/ou países. 

Sob essa perspectiva, o Curso de Engenharia Eletrônica da UTFPR Toledo busca atender às demandas locais ao oferecer conhecimento para ser aplicado no setor produtivo e social. Além disso, atrai e absorve múltiplas habilidades ao integrar estudantes de diversas localidades, promovendo expertises, geradas no processo de ensino, pesquisa e extensão. Esse processo ainda permite maior qualificação profissional e induz à inovação, atração de investimentos localmente e na região, que, por sua vez, gera melhor qualidade de vida e produção de riquezas.

\subsection{Valores UTFPR: integração social}

A formação em integração social diz respeito a realização de ações interativas com a sociedade para o desenvolvimento social e tecnológico. De acordo com o PDI 2018 - 2022 \cite{pdiutfpr}, os cursos da UTFPR, nos diferentes níveis e modalidades de ensino, deverão dar ênfase à ampla formação, que proporcione atitudes interativas e que valorize a atualização constante, promovendo estratégias e métodos de intervenção, cooperação, análise e reflexão, construindo um processo colaborativo e investigativo no âmbito da educação tecnológica, na vivência com os problemas reais da sociedade, voltados para o desenvolvimento sustentável, para a aplicação da tecnologia e para a busca de alternativas inovadoras para resolução de problemas.

Desde a sua criação, o curso tem participado sistematicamente de atividades da região, através de diversas pesquisas e trabalhos direcionados ao setor industrial e ao desenvolvimento tecnológico, ocupando posição de destaque e liderança na formação de recursos humanos. É esperado que o engenheiro seja capaz de avaliar os impactos sociais provocados pelo desenvolvimento tecnológico e identificar oportunidades de atuação para o benefício da sociedade. Não se limitam somente às disciplinas como meios para formação, conta-se também com projetos de extensão desenvolvidos no decorrer do curso que podem apresentar diversificados temas englobando como exemplo a sustentabilidade.  Além disso, o desenvolvimento de um projeto independente da área necessita que se coloque em prática a ética, que deve estar presente nas ações permitindo o contato dos estudantes com a mesma.
