\chapter{VALORES E PRINCÍPIOS INSTITUCIONAIS}

Conforme definido em seu Plano de Desenvolvimento Institucional (PDI), para o quadriênio 2018-2022 \cite{pdiutfpr}, a UTFPR apresenta como valores e princípios institucionais a sua missão, a sua visão e seus valores fundamentais descritos a seguir:

\nomenclature[A]{PDI}{Plano de Desenvolvimento Institucional}

\begin{itemize}[label={}]
	\item \textbf{Missão}: Desenvolver a educação tecnológica de excelência por meio do ensino, pesquisa e extensão, interagindo de forma ética, sustentável, produtiva e inovadora com a comunidade para o avanço do conhecimento e da sociedade;
	
	\item \textbf{Visão}: Ser modelo educacional de desenvolvimento social e referência na área tecnológica;
	
	\item \textbf{Valores fundamentais}:
	
	\begin{enumerate}
		\item Ética: gerar e manter a credibilidade junto à sociedade;
		
		\item Desenvolvimento Humano: formar o cidadão integrado no contexto social;
		\item Integração Social: realizar ações interativas com a sociedade para o desenvolvimento social e tecnológico;
		\item Inovação: efetuar a mudança por meio da postura empreendedor;
		\item Qualidade e Excelência: promover a melhoria contínua dos serviços oferecidos para a satisfação da sociedade;
		\item Sustentabilidade: assegurar que todas as ações se observem sustentáveis nas dimensões sociais, ambientais e econômicas.
		
	\end{enumerate}

\end{itemize}

\section{VALORES/PRINCÍPIOS ORIENTADORES DA GRADUAÇÃO}

A partir da sua missão e visão, a UTFPR estabeleceu a ética, o desenvolvimento humano, a integração social, a inovação, a qualidade e excelência e a sustentabilidade, como os valores fundamentais para a constituição dos princípios e da identidade das graduações.

Os cursos de graduação da UTFPR oferecem formação de recursos humanos para os diversos setores da sociedade, notadamente, os setores da economia envolvidos com práticas tecnológicas e os setores educacionais, a partir da vivência dos estudantes com os problemas reais da sociedade, em especial, àqueles relacionados ao desenvolvimento socioeconômico local e regional, às competências de padrão internacional, ao desenvolvimento e aplicação da tecnologia, e à busca de alternativas inovadoras para a resolução de problemas técnicos e sociais (\href{https://sei.utfpr.edu.br/sei/publicacoes/controlador_publicacoes.php?acao=publicacao_visualizar&id_documento=888276&id_orgao_publicacao=0}{Resolução COGEP 90/2018, art. 1\textordmasculine}) \nocite{cogep90}. 
