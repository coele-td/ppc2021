\chapter{ESTRUTURA ORGANIZACIONAL DO CURSO}

A estrutura organizacional do curso é composta pela Coordenação de curso, assessorada pelo Colegiado e o Núcleo Docente Estruturante (NDE).

\section{Coordenação do curso}

A Coordenação do Curso é exercida por um docente lotado no Departamento Engenharia Eletrônica, contratado em regime de tempo integral. O Coordenador de Curso é entendido no âmbito da Universidade como gestor pedagógico, do qual se espera o compromisso com o investimento na melhoria da qualidade do curso, analisando as dimensões didáticas, pedagógicas, administrativas e políticas, mediante o exercício da liderança ética, democrática e inclusiva, que se materialize em ações propositivas e proativas. Ressalta-se que a escolha do(a) coordenador(a) é norteada Resolução N\textordmasculine 145/2019 — COGEP, de 6 de dezembro de 2019 \cite{cogep145}.

A coordenação do curso é sempre exercida por um docente com formação e experiência na docência e preferencialmente na área, dedicando pelo menos 20 h semanais à atividade. Os horários de atendimento ao discente sempre considera o turno do curso.

As atribuições do coordenador constam no Regimento dos Campi da UTFPR. Seção VI. Subseção III — Das Coordenações de Curso, Arts. 27\textordmasculine, 28\textordmasculine e 29\textordmasculine{} \cite{regimento}. Além destas, o coordenador pode, por exemplo, propor em conjunto com os outros órgãos colegiados, mecanismos para a avaliação do desempenho do curso.

Dessa forma, a atuação do coordenador do Curso de Engenharia Eletrônica abrange:

\begin{itemize}
    \item Garantir o cumprimento das normas institucionais;
    \item Congregar e orientar os estudantes e atividades do curso, sob sua responsabilidade;
    \item Controlar e avaliar, em conjunto com o Colegiado do Curso, o desenvolvimento dos projetos pedagógicos e da ação didático pedagógica, no âmbito do curso;
    \item Coordenar a elaboração e divulgar à comunidade os planos de ensino das disciplinas do seu curso;
    \item Coordenar o processo de planejamento de ensino, no âmbito do curso;
    \item Coordenar a elaboração de propostas de alteração e atualização curricular do curso;
    \item Coordenar as atividades relacionadas aos componentes curriculares constantes nos projetos pedagógicos dos cursos;
    \item Propor cursos de formação continuada;
    \item Zelar pelas questões disciplinares dos estudantes;
    \item Acompanhar e orientar o docente nas questões didático-pedagógicas;
    \item Realizar com as outras coordenações e a Diretoria de Graduação e Educação Profissional à alocação dos docentes nas disciplinas;
    \item Coordenar as ações relacionadas ao reconhecimento e renovação de reconhecimento do curso;
    \item Solicitar e encaminhar os documentos acadêmicos, inclusive os de resultados de avaliações de ensino, nas datas estabelecidas no calendário acadêmico;
    \item Coordenar as atividades relacionadas com os processos de avaliação externa dos estudantes;
    \item Propor, com a anuência da Chefia nos termos da política institucional, a contratação dos docentes ou a alteração da jornada de trabalho destes;
    \item Participar, com a Chefia do Departamento Acadêmico, da avaliação de pessoal docente e administrativo, no âmbito do Departamento;
    \item Coordenar, em consonância com o Departamento Acadêmico, o processo de matrícula;
    \item Atuar na divulgação do curso; 
    \item promover a articulação entre as áreas de seu curso com outras Coordenações de Curso; e
    \item Controlar e avaliar o desempenho dos monitores, no âmbito do seu curso.
\end{itemize}

\section{Colegiado do curso}

O Colegiado de Curso é um órgão consultivo do curso para os assuntos de política de ensino, pesquisa e extensão, em conformidade como as diretrizes da UTFPR. O objetivo do Colegiado do Curso Engenharia Eletrônica é auxiliar a Coordenação do Curso visando à melhoria da qualidade do ensino, considerando os aspectos de infraestrutura, qualificação do corpo docente, atualizações do PPC e melhoria do desempenho do corpo discente. As atribuições do colegiado de curso constam na Resolução N\textordmasculine{} 103/2019 — COGEP — de 27 de novembro de 2019 \cite{cogep103}.

Dessa forma, São membros do colegiado do curso de Engenharia Eletrônica:

\begin{enumerate}
    \item Coordenador do Curso, na presidência;
    \item Professor responsável pela atividade de estágio - PRAE;
    \item Professor responsável pelo trabalho de conclusão de curso - PRATCC;
    \item Professor responsável pelas atividades de extensão - PRAExt;
    \item Professor responsável pelas atividades de internacionalização- PRAInt;
    \item Professor representante do colegiado de curso na Câmara Técnica do Conselho de Graduação e Educação Profissional (COGEP);
    \item \label{itm:especifica}Dois docentes eleitos pelos seus pares e seus respectivos suplentes que ministrem aulas ou tenham atividades relacionadas com as áreas específicas do curso de acordo com regras definidas por cada Coordenação no regulamento de eleição;
    \item Um docente eleito pelos seus pares ou indicado pelo coordenador de curso, que não se enquadre no item \ref{itm:especifica} e que ministre aulas no curso;
    \item Um representante discente, regularmente matriculado no curso, com seu respectivo suplente, indicado pelo órgão representativo dos alunos do curso, e na ausência deste, pelo Coordenador do Curso.
\end{enumerate}

Com isto, o colegiado dá voz a todos os professores diretamente envolvidos em lecionar os conteúdos específicos, enquanto integra professores de outras áreas. Adicionalmente, abre espaço tanto para os alunos que se encontram nos períodos iniciais, quanto para os que se encontram próximos de concluir o curso.

\section{Núcleo docente estruturante (NDE)}

O Núcleo Docente Estruturante — NDE — foi criado por meio da Portaria N\textordmasculine 147 do MEC de 2 de fevereiro de 2007 \cite{portaria147mec}, com o propósito de qualificar o envolvimento docente no processo de concepção e consolidação de um curso de graduação. As atribuições do NDE constam no Parecer CONAES N\textordmasculine 4 de 17 de junho de 2010 \cite{parecerconaes4} e respectiva Resolução N\textordmasculine 1 de 17 de junho de 2010, citada: ``O NDE de um curso de graduação é constituído por um grupo de docentes, com atribuições acadêmicas de acompanhamento, atuante no processo de concepção, consolidação e contínua atualização do PPC'' (BRASIL, 2010a). Ressalta-se que a atuação do NDE é um critério considerado pelo INEP na avaliação institucional e de cursos. Conforme a Resolução supracitada, Art. 2, são atribuições do Núcleo Docente Estruturante, entre outras: 