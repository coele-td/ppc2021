\chapter{ESTRUTURA ORGANIZACIONAL DO CURSO}

A estrutura organizacional do curso é composta pela Coordenação de curso, assessorada pelo Colegiado e Núcleo Docente Estruturante (NDE).

\section{Coordenação do curso}

A Coordenação do Curso é exercida por um docente lotado no Departamento Engenharia Eletrônica, contratado em regime de tempo integral. O Coordenador de Curso é entendido no âmbito da Universidade como gestor pedagógico, do qual se espera o compromisso com o investimento na melhoria da qualidade do curso, analisando as dimensões didáticas, pedagógicas, administrativas e políticas, mediante o exercício da liderança ética, democrática e inclusiva, que se materialize em ações propositivas e proativas. Ressalta-se que a escolha do(a) coordenador(a) é norteada pela Resolução N\textordmasculine 145/2019 — COGEP, de 6 de dezembro de 2019 \cite{cogep145}.

A coordenação do curso é sempre exercida por um docente com formação e experiência na docência e preferencialmente na área, dedicando pelo menos 20 h semanais à atividade. Os horários de atendimento ao discente sempre considera o turno do curso.

As atribuições do coordenador constam no Regimento dos Campi da UTFPR. Seção VI. Subseção III — Das Coordenações de Curso, Arts. 27\textordmasculine, 28\textordmasculine e 29\textordmasculine{} \cite{regimento}. Além destas, o coordenador pode, por exemplo, propor em conjunto com os outros órgãos colegiados, mecanismos para a avaliação do desempenho do curso.

Dessa forma, a atuação do coordenador do Curso de Engenharia Eletrônica abrange:

\begin{itemize}
    \item Garantir o cumprimento das normas institucionais;
    \item Congregar e orientar os estudantes e atividades do curso, sob sua responsabilidade;
    \item Controlar e avaliar, em conjunto com o Colegiado do Curso, o desenvolvimento dos projetos pedagógicos e da ação didático pedagógica, no âmbito do curso;
    \item Coordenar a elaboração e divulgar à comunidade os planos de ensino das disciplinas do seu curso;
    \item Coordenar o processo de planejamento de ensino, no âmbito do curso;
    \item Coordenar a elaboração de propostas de alteração e atualização curricular do curso;
    \item Coordenar as atividades relacionadas aos componentes curriculares constantes nos projetos pedagógicos dos cursos;
    \item Propor cursos de formação continuada;
    \item Zelar pelas questões disciplinares dos estudantes;
    \item Acompanhar e orientar o docente nas questões didático-pedagógicas;
    \item Realizar com as outras coordenações e a Diretoria de Graduação e Educação Profissional (DIRGRAD) a alocação dos docentes nas disciplinas;
    \item Coordenar as ações relacionadas ao reconhecimento e renovação de reconhecimento do curso;
    \item Solicitar e encaminhar os documentos acadêmicos, inclusive os de resultados de avaliações de ensino, nas datas estabelecidas no calendário acadêmico;
    \item Coordenar as atividades relacionadas com os processos de avaliação externa dos estudantes;
    \item Propor, com a anuência da Chefia nos termos da política institucional, a contratação dos docentes ou a alteração da jornada de trabalho destes;
    \item Participar, com a Chefia do Departamento Acadêmico, da avaliação de pessoal docente e administrativo, no âmbito do Departamento;
    \item Coordenar, em consonância com o Departamento Acadêmico, o processo de matrícula;
    \item Atuar na divulgação do curso; 
    \item promover a articulação entre as áreas de seu curso com outras Coordenações de Curso; e
    \item Controlar e avaliar o desempenho dos monitores, no âmbito do seu curso.
\end{itemize}

\section{Colegiado do curso}

O Colegiado de Curso é um órgão consultivo do curso para os assuntos de política de ensino, pesquisa e extensão, em conformidade como as diretrizes da UTFPR. O objetivo do Colegiado do Curso Engenharia Eletrônica é auxiliar a Coordenação do Curso visando a melhoria da qualidade do ensino, considerando os aspectos de infraestrutura, qualificação do corpo docente, atualizações do PPC e melhoria do desempenho do corpo discente. As atribuições do colegiado de curso constam na Resolução N\textordmasculine{} 103/2019 — COGEP — de 27 de novembro de 2019 \cite{cogep103}.

Dessa forma, São membros do colegiado do curso de Engenharia Eletrônica:

\begin{enumerate}
    \item Coordenador do Curso, na presidência;
    \item Professor responsável pela atividade de estágio - PRAE;
    \item Professor responsável pelo trabalho de conclusão de curso - PRATCC;
    \item Professor responsável pelas atividades de extensão - PRAExt;
    \item Professor responsável pelas atividades de internacionalização- PRAInt;
    \item Professor representante do colegiado de curso na Câmara Técnica do Conselho de Graduação e Educação Profissional (COGEP);
    \item \label{itm:especifica}Dois docentes eleitos pelos seus pares e seus respectivos suplentes que ministrem aulas ou tenham atividades relacionadas com as áreas específicas do curso de acordo com regras definidas por cada Coordenação no regulamento de eleição;
    \item Um docente eleito pelos seus pares ou indicado pelo coordenador de curso, que não se enquadre no item \ref{itm:especifica} e que ministre aulas no curso;
    \item Um representante discente, regularmente matriculado no curso, com seu respectivo suplente, indicado pelo órgão representativo dos alunos do curso, e na ausência deste, pelo Coordenador do Curso.
\end{enumerate}

O processo de escolha dos membros do colegiado respeita a SEÇÃO VI - Das eleições para composição do colegiado, da Resolução N\textordmasculine{} 103/2019 — COGEP \cite{cogep103}. As ações do Colegiado são descritas no Artigo 3º desta mesma resolução. A frequência de reuniões é descrita na SEÇÃO VII - DAS REUNIÕES. Os demais procedimentos podem ser consultados no próprio Regulamento.

Por fim, pode-se concluir que o colegiado dá voz a todos os professores diretamente envolvidos em lecionar os conteúdos específicos, enquanto integra professores de outras áreas e os acadêmicos. Neste sentido, permite a participação de todos no processo de avaliação das propostas elaboradas pelo Núcleo Docente Estruturante (NDE) e contribuição com a construção do curso.


\section{Núcleo docente estruturante (NDE)}

O NDE constitui-se de um grupo de docentes, com atribuições acadêmicas de acompanhamento, atuante no processo de concepção, avaliação, solidificação e contínua atualização do Projeto Pedagógico do Curso (PPC). Este grupo é caracterizado por ser responsável pela formulação, implementação e desenvolvimento do PPC. O núcleo Foi criado por meio da Portaria N\textordmasculine 147 do MEC de 2 de fevereiro de 2007 \cite{portaria147mec}, com o propósito de qualificar o envolvimento docente no processo de concepção e consolidação de um curso de graduação. As atribuições do NDE constam no Parecer CONAES N\textordmasculine 4 de 17 de junho de 2010 \cite{parecerconaes4} e respectiva Resolução N\textordmasculine 1 de 17 de junho de 2010 \cite{resconaes1}, citada: ``O NDE de um curso de graduação é constituído por um grupo de docentes, com atribuições acadêmicas de acompanhamento, atuante no processo de concepção, consolidação e contínua atualização do PPC''. Ressalta-se que a atuação do NDE é um critério considerado pelo INEP na avaliação institucional e de cursos. Conforme a Resolução supracitada, Art. 2\textordmasculine, são atribuições do Núcleo Docente Estruturante: 

\begin{enumerate}
    \item contribuir para a consolidação do perfil profissional do egresso do curso;
    \item zelar pela integração curricular interdisciplinar entre as diferentes atividades de ensino constantes no currículo;
    \item zelar pelo cumprimento das Diretrizes Curriculares Nacionais para os Cursos de Graduação;
    \item zela pelo cumprimento das Diretrizes Curriculares Nacionais para os cursos de Graduação.
\end{enumerate}

 No ambito institucional, a Resolução N\textordmasculine 9/12-COGEP, de 13 de abril de 2012 \cite{cogep9} institui o Regulamento Do Núcleo Docente Estruturante Dos Cursos De Graduação da UTFPR.

O NDE do Curso de Graduação em Engenharia Eletrônica do Campus Toledo é composto, em sua maioria, por docentes da área específica do curso de Engenharia Eletrônica, podendo existir membros das áreas de matemática, física, computação e humanidades. 

O NDE é responsável pela manutenção de um currículo atualizado com as novas tecnologias e tendências de ensino, bem como o constante aperfeiçoamento do PPC. Para isso, o NDE se reúne periodicamente para elaboração e mantém discussões sobre os assuntos de cunho pedagógico relacionados diretamente ao curso. Este propõe alterações para integração curricular interdisciplinar, buscando sempre por melhorias e incentivos a pesquisa e extensão e zelando pelo cumprimento das diretrizes curriculares.

%O NDE reúne-se periodicamente para elaboração e constante aperfeiçoamento do PPC. Mantém discussões sobre os assuntos de cunho pedagógico relacionados diretamente ao curso. Propõe alterações para integração curricular interdisciplinar. Discute melhorias e incentivos a pesquisa e extensão, sempre zelando pelo cumprimento das diretrizes curriculares. Mantém um currículo atualizado com as novas tecnologias e tendências do ensino.

\section{Corpo docente}

O corpo docente da UTFPR, por ser uma universidade oriunda do antigo CEFET-PR, é constituído por Professores do Ensino Básico, Técnico e Tecnológico (EBTT), Professores de Magistério Superior, e, eventualmente, por Professores Visitantes e Professores Substitutos.

Os documentos institucionais Regimento Geral da UTFPR e Estatuto da UTFPR \cite{estatutoutfpr} se referem ao corpo docente no Título V, Capítulo I. As atividades docentes relacionadas ao Ensino, Pesquisa e Extensão estão definidas no Regulamento da Atividade Docente da Universidade Tecnológica Federal do Paraná, conforme DELIBERAÇÃO COUNI N\textordmasculine 25/2018, de 14 de setembro de 2018 \cite{couni25}. 

A composição do corpo do docente das áreas profissionalizante e profissionalizante específica do curso de Engenharia Eletrônica, considerando formação na graduação, titulação acadêmica stricto senso e regime de trabalho, é apresentada nos Quadros \ref{qua:profs} e \ref{qua:titulacao}.

    \begin{quadro}
        \centering
        \caption[Corpo Docente]{Composição e formação do corpo docente da COELE-TD das áreas profissionalizante e profissionalizante específica (em Novembro de 2021)}        
        \label{qua:profs}
        \begin{tabularx}{\textwidth}{| >{\centering\arraybackslash\small}X |> {\centering\arraybackslash\small}X |>{\centering\arraybackslash\small}X |>{\centering\arraybackslash\small}X |}
            \toprule%
            \rowcolor{white}\bfseries Docente & \bfseries Graduação & \bfseries Titulação & \bfseries Regime dee Trabalho\\
            \midrule
            \csvreader[	head to column names,
                        late after line=\csvifoddrow{\\}{\\\rowcolor{gray!10}} \hline, 
                        separator=pipe]%
                        {Caps/Quadros/corpodocente.csv}{}%
                        {\docente & \graduacao & \titulacao & \regime}%
            %\bottomrule
            \end{tabularx}
    \end{quadro}

    \begin{table}
        \centering
        \caption[Titulação do Corpo Docente]{Titulação do corpo docente da COELE-TD (em Novembro de 2021)}        
        \label{qua:titulacao}
        \begin{tabularx}{\textwidth}{ >{\centering\arraybackslash\small}X >{\centering\arraybackslash\small}X >{\centering\arraybackslash\small}X}
            \toprule%
            \rowcolor{white}\bfseries Especialistas & \bfseries Mestres & \bfseries Doutores \\
            \midrule
            \csvreader[	head to column names,
                        late after line=\csvifoddrow{\\}{\\\rowcolor{gray!10}}, 
                        separator=pipe]%
                        {Caps/Quadros/titulacao.csv}{}%
                        {\Especialistas & \Mestres & \Doutores }%
            \bottomrule
            \end{tabularx}
    \end{table}

