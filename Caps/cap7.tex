\chapter{ESTRUTURA ORGANIZACIONAL DO CURSO}

A estrutura organizacional do curso é composta pela Coordenação de curso, assessorada pelo Colegiado e o Núcleo Docente Estruturante (NDE).

\section{Coordenação do curso}

A Coordenação do Curso é exercida por um docente lotado no Departamento Engenharia Eletrônica, contratado em regime de tempo integral. O Coordenador de Curso é entendido no âmbito da Universidade como gestor pedagógico, do qual se espera o compromisso com o investimento na melhoria da qualidade do curso, analisando as dimensões didáticas, pedagógicas, administrativas e políticas, mediante o exercício da liderança ética, democrática e inclusiva, que se materialize em ações propositivas e proativas. Ressalta-se que a escolha do(a) coordenador(a) é norteada Resolução N\textordmasculine 145/2019 — COGEP, de 6 de dezembro de 2019 \cite{cogep145}.

A coordenação do curso é sempre exercida por um docente com formação e experiência na docência e preferencialmente na área, dedicando pelo menos 20 h semanais à atividade. Os horários de atendimento ao discente sempre considera o turno do curso.

As atribuições do coordenador constam no Regimento dos Campi da UTFPR. Seção VI. Subseção III — Das Coordenações de Curso, Arts. 27\textordmasculine, 28\textordmasculine e 29\textordmasculine{} \cite{regimento}. Além destas, o coordenador pode, por exemplo, propor em conjunto com os outros órgãos colegiados, mecanismos para a avaliação do desempenho do curso.

\section{Colegiado do curso}

O Colegiado de Curso é um órgão consultivo do curso para os assuntos de política de ensino, pesquisa e extensão, em conformidade como as diretrizes da UTFPR. O objetivo do Colegiado do Curso Engenharia Eletrônica é auxiliar a Coordenação do Curso visando à melhoria da qualidade do ensino, considerando os aspectos de infraestrutura, qualificação do corpo docente, atualizações do PPC e melhoria do desempenho do corpo discente. As atribuições do colegiado de curso constam na Resolução N\textordmasculine{} 103/2019 — COGEP — de 27 de novembro de 2019 \cite{cogep103}.