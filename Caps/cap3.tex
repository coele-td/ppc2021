\chapter{POLÍTICAS DE ENSINO}
\label{chap:politicas}

Na estruturação de seu PDI 2018-2022 \cite{pdiutfpr} a UTFPR estabeleceu como princípios norteadores para as políticas de seus cursos de graduação a flexibilidade curricular, a articulação com a sociedade, a mobilidade acadêmica, a sustentabilidade, a interculturalidade, a inovação curricular e metodológica e a internacionalização.

Somado a isso as Diretrizes Curriculares dos Cursos de Graduação da UTFPR \cite{cogep90} dão centralidade à sustentabilidade, ao empreendedorismo, à superação do currículo segmentado, ampliando assim a flexibilidade curricular e a proposição de cursos de caráter inovador.

Portanto, a elaboração deste Projeto Pedagógico de Curso está pautada na reformulação do Bacharelado em Engenharia Eletrônica da UTFPR, Câmpus Toledo, com base nos documentos institucionais vigentes, PDI \cite{pdiutfpr} e PPI \cite{ppiutfpr}, bem como em consonância com vertentes contemporâneas de educação em Engenharia no Brasil e no mundo.

A reformulação parte das habilidades e atitudes a serem desenvolvidas pelos discentes segundo a construção de competências do curso de Engenharia, compatíveis com os valores e princípios institucionais, com as Diretrizes Curriculares Nacionais para os Cursos de Engenharia – CNE/CES e com os documentos normativos do conselho de classe – CONFEA/CREA.

Para que o perfil profissional do egresso pretendido pelo Curso de Engenharia Eletrônica seja obtido, a instituição, em conjunto com o curso, proporá práticas pedagógicas para a condução do currículo, visando estabelecer as dimensões investigativa e interativa como princípios formativos e condição central da formação profissional e da relação teoria e realidade. As políticas institucionais promovidas pela UTFPR, e adotadas, de forma direta, no Curso de Engenharia Eletrônica são descritas a seguir.

\section{ARTICULAÇÃO ENTRE A TEORIA E A PRÁTICA E INTERDISCIPLINARIDADE}
\label{sec:artc}

Ao longo de toda sua história, a UTFPR sempre teve o compromisso de romper com a dualidade entre teoria e prática, dimensões estas indissociáveis para a educação integral do indivíduo, pois nenhuma atividade humana se realiza sem elaboração mental, sem uma teoria em que se referencie e lhe dê sustentação. Tal princípio educativo não admite a separação entre as funções intelectuais e as técnicas e respalda uma concepção de formação profissional que unifique ciência, tecnologia e trabalho, bem como atividades intelectuais e instrumentais (\pdfmarkupcomment{UNIVERSIDADE, 2013a}{Esse é o PDI antigo (2013-2017), precisa verificar se o texto acompanha o PDI atual (2018-2022)}).

A educação em todos os seus níveis e modalidades deve ser encarada como referencial permanente de formação geral, que encerra como objetivo fundamental o desenvolvimento do ser humano pautado por valores éticos, sociais e políticos, de maneira a preservar a sua dignidade e a desenvolver ações junto à sociedade com base nos mesmos valores. Assim, dentro da carga horária de cada disciplina do curso são desenvolvidos os pressupostos teóricos necessários juntamente com as práticas experimentais/laboratoriais/de campo pertinentes ao conteúdo desenvolvido, utilizando-se dos espaços necessários disponíveis na estrutura do campus.

%\pdfmarkupcomment{As disciplinas que preveem atividades práticas de laboratório têm na ordem de no mínimo 25\% (vinte e cinco por cento) de sua carga horária total voltada para este fim, podendo chegar a 50\% (cinquenta por cento), e estas atividades deverão ser explicitadas nos respectivos Planos de Ensino. As aulas práticas são destinadas a turmas de até 22 (vinte e dois) alunos, ocorrendo desdobramento de turmas quando o número de matrículas excederem esse quantitativo. Nesses casos, a carga horária docente e o desenvolvimento das atividades experimentais são contabilizados e realizados em dobro.}{Essa parte precisamos trabalhar. Verificar se é possível fazer isso durante a matrícula. E se teremos professores suficientes para bancar laboratórios replicados.}

De acordo com a filosofia de concepção de curso, com base nos regimentos internos da UTFPR e nas Resoluções do CONFEA/CREA, a prática acompanhará a teoria com carga horária definida para conteúdos profissionalizantes específicos. Adicionalmente, uma determinada carga horária prática também é contemplada nos núcleos de conteúdos básicos e profissionalizantes, buscando, sempre que possível, a mesma carga horária.

Em sendo assim, cada disciplina definida no curso, apresenta cargas horárias semanais definidas entre Aulas Teóricas – AT e Aulas Práticas – AP, desta forma compondo a carga horária total da disciplina. A integração entre teoria e prática fica a cargo do professor que ministrará a unidade curricular a partir de atividades de campo e/ou laboratoriais, simulações, estudos de caso, projetos, dentre outras, conforme sua escolha. Estas atividades serão sistematizadas de acordo com os seguintes pontos balizadores:

\begin{itemize}
    \item Apresentar problemas em situações reais e/ou simuladas visando a aproximação do discente com contextos reais da engenharia;
    \item Motivar o aluno por meio da aplicação prática de conteúdos trabalhados;
    \item Integrar teoria e prática para melhor compreensão e assimilação dos temas de estudo, destacando o caráter indissociável;
    \item Ser instrumento de avaliação do aluno.
\end{itemize}

Além disso, o curso conta com um instrumento de Atividade Prática chamada de Disciplina Integradora que possui caráter interdisciplinar, uma vez que articula fundamentos técnicos da engenharia eletrônica aliados a gestão de projetos e ao desenvolvimento de habilidades pessoais e interpessoais. Neste processo, todas as competências trabalhadas durante o curso são integradas em períodos específicos. % Análise de circuitos (Básica), Medidas e sensores (analisar e avaliar) e Sistemas embarcados (Criar)

A partir do 7º período, o aluno pode realizar estágio curricular obrigatório, conforme estabelece as Diretrizes Curriculares Nacionais \cite{dcneng}. Neste contexto, o estudante é capaz de colocar em prática todo o ensinamento recebido durante seus anos de estudo no curso, sendo acompanhado por um professor orientador e um supervisor responsável pelo estágio na empresa que o oferece.

Cabe salientar que, o Estágio Curricular Supervisionado deve fornecer condições suficientes para que o aluno possa, de acordo com o PDI, inserir-se com maior facilidade no mercado de trabalho. O PDI destaca: “o estágio curricular é obrigatório para todos os cursos de nível técnico e de graduação, visa à complementação do processo ensino-aprendizagem e tem como objetivos: (i) facilitar a futura inserção do estudante no mundo de trabalho; e (ii) facilitar a adaptação social e psicológica do estudante à futura atividade profissional” \cite{pdiutfpr}.

Adicionalmente, considera-se que o estágio merece destaque por se constituir como espaço privilegiado de aprendizagem, que permite ao estudante integrar-se ao mundo do trabalho, deparando-se com situações, relacionamentos, técnicas e posturas do ambiente profissional que enriquecem e complementam sua formação acadêmico-profissional e empreendedora.

\section{DESENVOLVIMENTO DE COMPETÊNCIAS PROFISSIONAIS}
\label{sec:desen}

%Entende-se por Competências Profissionais, a possibilidade, para um indivíduo, de mobilizar de maneira interiorizada um conjunto integrado de recursos em vista de resolver uma família de situações-problema \cite{scallon2017}. 

%\pdfmarkupcomment{Completar}{O texto aqui precisa seguir as novas DCNs, o texto do PPC vigente (3.2) segue o disposto na DCN antiga.}

Os cursos de graduação da UTFPR, de acordo com o item 3.3.2 Desenvolvimento de competências profissionais do PDI, propõem o desenvolvimento de competências profissionais, entendidas como:

\begin{citacao}
    {[...]} por sua natureza e suas características, a educação profissional e tecnológica deve contemplar o desenvolvimento de competências gerais e específicas, incluindo fundamentos científicos e humanísticos necessários ao desempenho profissional e à atuação cidadã \cite{pdiutfpr}.
\end{citacao}

Primeiramente é pertinente estabelecer que o conceito de competência assumido se refere “a possibilidade, para um indivíduo, de mobilizar de maneira interiorizada um conjunto integrado de recursos em vista de resolver uma família de situações-problema” \cite{scallon2017}. As competências, sejam gerais ou específicas, são desenvolvidas por meio de processos educativos estabelecidos na organização do ensino no curso, envolvendo:

\begin{itemize}
    \item utilização de métodos diferenciados de ensino e novas formas de organização do trabalho acadêmico, que propiciem o desenvolvimento de capacidades para resolver problemas que integram a vivência e a prática profissional;
    \item incorporação dos saberes dos estudantes às práticas de ensino, como forma de reconhecimento de possibilidades de soluções de problemas, assim como de percursos de aprendizagem;
    \item estímulo à criatividade, à autonomia intelectual e ao empreendedorismo;
    \item valorização das inúmeras relações entre conteúdo e contexto, que se podem estabelecer;
    \item integração de estudos de diferentes campos, como forma de romper com a segmentação e o fracionamento, entendendo que os conhecimentos se inter-relacionam, contrastam-se, complementam-se, ampliam-se e influenciam uns nos outros \cite{pdiutfpr}.
\end{itemize}

As competências profissionais são desenvolvidas pelos discentes em todas as disciplinas do curso, iniciando em nível cognitivo baixo até, ao final, alcançarem o nível cognitivo mais alto. Assim, as competências são desenvolvidas gradativamente em várias unidades curriculares e em momentos distintos no decorrer do curso. 

\section{FLEXIBILIDADE CURRICULAR}
\label{sec:flex}

A flexibilidade curricular considera uma construção de curso baseada na diminuição de pré-requisitos, na oferta de diversos caminhos formativos, de disciplinas optativas e disciplinas eletivas, assim como a facilitação da mobilidade acadêmica.

A flexibilização curricular, assegurada pelo PNE 2014-2024, \href{http://www.planalto.gov.br/ccivil_03/_ato2011-2014/2014/lei/l13005.htm}{Lei n° 13.005/2014} \cite{Lei:13005:2014}, é fundamental para atender a demanda social por profissionais que compreendam as novas relações de produção, de trabalho e suas exigências, a demanda pelo conhecimento articulado a produção do saber e de novas tecnologias, a demanda por formação crítica e de profissionais competentes \cite{pdiutfpr}.

Baseada na indissociabilidade entre ensino, pesquisa e extensão, a flexibilização curricular possibilita, por percursos formativos diferenciados, a formação de profissionais competentes, com domínio de habilidades técnicas e cognitivas, com apropriação científica sólida. Os percursos formativos diferenciados rompem com o enfoque unicamente unidade curricular e sequenciado e permitem aos alunos novas formas de apreensão e integração de conhecimentos. Nessa perspectiva, o estudante pode ampliar os horizontes do conhecimento, é capaz de uma visão crítica que lhe permite extrapolar a aptidão específica de seu campo de atuação profissional.

A flexibilização curricular deve possibilitar ao estudante percursos formativos diferenciados para construção das mesmas competências, permitindo inclusive a participação do estudante nas escolhas desses percursos formativos, de ambientes diferenciados de ensino, proporcionando aos discentes visão crítica que lhe permite extrapolar a aptidão específica de seu campo de atuação profissional, estimulando a aprendizagem permanente, a formação de competências e o domínio de habilidades técnicas e cognitivas desejadas.

%\pdfmarkupcomment{É consensual a constatação de estarem superados os modelos de ensino estruturados sob a ótica de grades curriculares inflexíveis, estanques, caracterizadas pela fragmentação e hierarquização rígida das disciplinas (UNIVERSIDADE, 2013a; UNIVERSIDADE,2007a).}{O Alberto V. sugeriu a remoção dete parágrafo, assim como eu sugiro}

Fica evidente, através do PPI \cite{ppiutfpr} e PDI \cite{pdiutfpr}, que a UTFPR tem o compromisso de garantir estruturas curriculares mais inovadoras e flexíveis, permitindo que o aluno tenha participação no ritmo e na direção do seu curso, utilizando-se da melhor forma os mecanismos que a Universidade oferece em termos de atividades acadêmicas na composição de seu currículo.

Este compromisso institucional atende não somente a \href{http://www.planalto.gov.br/ccivil_03/leis/leis_2001/l10172.htm}{Lei n° 10.172/2001} \cite{Lei:10172:2001} e os Pareceres CNE-CES n° 776/97 \cite{parecer776} e n° 583/01 \cite{parecer583}, mas também tem vistas à internacionalização, com medidas que venham a contribuir na flexibilidade dos currículos. A forma como os pré-requisitos são considerados institucionalmente, e a possibilidade de convalidação de disciplinas em bloco ou por saberes e competências são as principais ações que permitem aos cursos considerar, para integralização do currículo do discente, alternativas pessoais e percursos acadêmicos diferenciados.

A proposta é que se permita que várias atividades acadêmicas, que hoje já são desenvolvidas pelo estudante durante sua permanência na universidade, sejam contabilizados no seu histórico escolar. Neste cenário, o curso apresenta duas modalidades de flexibilização curricular: \textbf{vertical} e \textbf{horizontal}. 

A flexibilização vertical é realizada pela organização das disciplinas ao longo de semestres compreendendo o núcleo de formação específica. Ademais, as disciplinas são preferencialmente alocadas em turnos (manhã e tarde) alternados entre cada semestre. Dessa forma o aluno tem a oportunidade de adiantar uma disciplina do próximo semestre e assim concluir o curso em menos de cinco anos, ou então, um aluno que não obteve a aprovação em uma disciplina, pode cursá-la sem necessidade de deixar de cursar as disciplinas do semestre em que se encontra.

O núcleo específico é composto das disciplinas do núcleo comum, cursadas por todos os discentes do curso e das disciplinas optativas das áreas de aprofundamento escolhida pelo aluno, previstas na matriz curricular. Para as disciplinas optativas das áreas de aprofundamento e o Trabalho de Conclusão de Curso (TCC), foi definido um pré-requisito baseado no período, paralelamente aos pré-requisitos baseados em disciplinas. Ou seja, o aluno estará apto a cursar qualquer uma das disciplinas optativas ou TCC desde que esteja matriculado pelo menos no 7\textordmasculine{} período do curso, ou que possua as disciplinas pré-requisitos estabelecidas. Esses pré-requisitos foram estabelecidos porque são imprescindíveis ao bom rendimento escolar.

\nomenclature[A]{TCC}{Trabalho de Conclusão de Curso}

O núcleo não-específico é composto das disciplinas que não constam da matriz curricular do curso escolhido pelo aluno, mas que constituem seus interesses para complementar sua formação em outras áreas de interface, constituindo, assim, um percurso interdisciplinar. \pdfmarkupcomment{Esse núcleo é baseado em uma opção livre, em que o aluno pode cursar um determinado número de disciplinas fora da sua habilitação, sem aprovação prévia dos colegiados e de uma formação complementar realizada em outros cursos, com autorização dos colegiados ou de um número determinado de disciplinas da própria UTFPR que não constam do currículo do aluno ou que são excedentes nos grupos de sua habilitação}{Revisar, pois a frase está contraditória}. Essa escolha fica a critério do aluno. 

A flexibilização horizontal curricular de núcleo não-específico é realizada na UTFPR por meio da modalidade de enriquecimento curricular prevista no Artigo 28 do Regulamento da organização didático-pedagógica dos cursos de graduação da UTFPR \cite{rodp}. Esse artigo permite que o aluno possa cursar uma disciplina que não pertence ao seu curso. Nesse caso, o aluno fica dispensado da exigência de cumprimento dos pré-requisitos. O discente também tem a liberdade de cursar a disciplina de enriquecimento curricular no seu campus de origem ou em qualquer um dos campi da UTFPR, ou mesmo, em instituições com as quais exista acordo de mobilidade e/ou de dupla diplomação.

A flexibilização horizontal é implementada por meio da \pdfmarkupcomment{disciplina Atividades Complementares}{Verificar se será mantido as atividades complementares}, para que o aluno obtenha conhecimentos adicionais ao curso. Através de atividades ligadas à projetos de extensão, projetos de iniciação científicas, monitoria de disciplinas, línguas estrangeiras, informática, esportes, artes, e de acordo com o seu perfil pessoal, o estudante poderá complementar a sua formação, além de exercitar as atitudes esperadas incentivando-o a interagir com a sociedade em projetos sociais e acadêmicos.

\section{MOBILIDADE ACADÊMICA E INTERNACIONALIZAÇÃO}

A mobilidade acadêmica na instituição está prevista em dois planos: o interno (inter campi) e o externo (interuniversitário nacional e internacional).

O plano externo ocorre por meio de convênios mantidos pela UTFPR com Instituições Nacionais e Internacionais, incluído a dupla diplomação, conforme disposto nos Artigos 7\textordmasculine, 9\textordmasculine, 10\textordmasculine, 11\textordmasculine, 16\textordmasculine, 24\textordmasculine, 27\textordmasculine e 28\textordmasculine, do Regulamento da Organização Didático-Pedagógica dos cursos de Graduação da UTFPR \cite{rodp}.

O Programa de Mobilidade estudantil foi estruturado no campus Toledo com o propósito de aprimorar as atividades de ensino e pesquisa, propiciando a estudantes, docentes e funcionários da UTFPR a vivência de outras culturas e diferentes formas de aprendizagem.

Neste contexto, o Programa de Mobilidade Estudantil (PME) da UTFPR tem como objetivo propiciar a mobilidade acadêmica de estudantes regularmente matriculados em cursos de graduação. Todos os programas de mobilidade são de responsabilidade da DIRINTER – Diretoria de Relações Interinstitucionais.

\nomenclature[A]{PME}{Programa de Mobilidade Estudantil}
\nomenclature[A]{DIRINTER}{Diretoria de Relações Interinstitucionais}

Por mobilidade acadêmica entende-se o processo que possibilita o afastamento temporário ao estudante matriculado em uma Instituição de Ensino Superior (IES) para estudar em outra, prevendo que a conclusão do curso se dê na instituição de origem. Ademais, o PME da UTFPR é regido por regulamento próprio e abrange a Mobilidade Estudantil Nacional (MEN) e a Internacional (MEI).

\nomenclature[A]{IES}{Instituição de Ensino Superior}
\nomenclature[A]{MEN}{Mobilidade Estudantil Nacional}
\nomenclature[A]{MEI}{Mobilidade Estudantil Internacional}

\subsection{MOBILIDADE ESTUDANTIL NACIONAL}

A MEN alcança somente estudantes da UTFPR regularmente matriculados em cursos de graduação e os de Instituições Federais de Ensino Superior brasileiras e/ou de Instituições de Ensino Superior do estado do Paraná. Os critérios de elegibilidade dependem do que for estabelecido em edital, mas, em geral, a exigência é que os alunos já tenham cursado e concluído, no mínimo, vinte por cento da carga horária de integralização do curso de origem, bem como tenham, no máximo, duas reprovações acumuladas nos dois períodos letivos que antecedem o pedido de mobilidade.
 
O MEN tem por objetivo promover o intercâmbio entre estudantes da UTFPR e de Universidades Federais e das Estaduais Paranaenses conveniadas, proporcionando-lhes a possibilidade de ampliar seus conhecimentos através da vivência em outras Instituições de Ensino Superior.

Ressalta-se que a Mobilidade Acadêmica não é transferência de Instituição nem de curso.

\subsection{MOBILIDADE ESTUDANTIL INTERNACIONAL}

O programa de cooperação internacional teve início em 1958 com os Estados Unidos, para a implementação do Centro de Formação de Professores da \pdfmarkupcomment{CBAI}{não achei o significado da sigla}. Mais tarde, em 1989, a UTFPR firmou convênio com a Fachhochschule de Munique, na Alemanha.

Nos últimos anos várias instituições alemãs têm mantido intercâmbio de estudantes, possibilitando que alemães estudem e estagiem no Brasil, do mesmo modo que estudantes brasileiros na Alemanha. Houve um crescimento também da preferência pelas universidades de tecnologia francesas. Hoje, além de Alemanha e França, a UTFPR busca ampliar a cooperação acadêmica com outros países tanto no continente europeu quanto americano e africano.

O curso de Engenharia Eletrônica da UTFPR campus Toledo firmou convênio de Dupla Diplomação com o Instituto Politécnico de Bragança (IPB) de Portugal em 2016 e já teve \pdfmarkupcomment{6 (seis) alunos}{atualizar} enviados por esse convênio. \pdfmarkupcomment{Recentemente, no final de 2017, o curso também firmou convênio com a Université de Technologie de Compiègne (UTC) da França.}{Corrigir, o convenio é com a UTFPR como um todo}

\nomenclature[A]{IPB}{Instituto Politécnico de Bragança}

\section{ARTICULAÇÃO COM A PESQUISA E PÓS GRADUAÇÃO}

A UTFPR entende a Pesquisa, a Iniciação Científica, a Inovação Tecnológica, Artística e Cultural como um conjunto de ações que visam a descoberta de novos conhecimentos, consistindo-se em um dos pilares da atividade acadêmica. Pesquisar implica distanciar-se da reprodução acrítica de práticas tradicionais, requer por em jogo processos reflexivos nos quais a interação social e as atividades metacognitivas se fortalecem. Uma visão da investigação como esta é, portanto, um instrumento potente para orientar e favorecer o avanço da ciência e o desenvolvimento profissional \cite{pizzatoconcepccoes}.

O ensino e a pesquisa de forma indissociável colaboram para viabilizar a relação transformadora entre a universidade e a sociedade. Desenvolver projetos de pesquisas que acolham estudantes em diferentes estágios formativos, apoiados nos grupos de estudos e no uso comum da infraestrutura disponível colabora para tanto. A articulação do ensino com as iniciativas de pesquisa e pós-graduação deve considerar o compromisso da instituição com as principais questões e desafios da sociedade, como elemento importante para dupla conscientização, a saber: a do pesquisador ao aceitarem também como desafio acadêmico a busca de soluções para problemas reais; e da sociedade de um modo geral, e do mundo do trabalho em particular, que poderá se beneficiar dos conhecimentos disponibilizados por iniciativas necessariamente submetidas às exigências decorrentes do ``rigor acadêmico''. Para que esse compromisso institucional seja mais efetivo, torna-se importante o esforço de exteriorizar, por um lado, o seu potencial de geração de novos conhecimentos e, por outro lado, o seu desejo que eles sejam compartilhados e aplicados como meio da promoção do desenvolvimento sustentável da região.

O curso de Engenharia Eletrônica da UTFPR, campus Toledo, tem como uma de suas prioridades as atividades de pesquisa, tanto em relação ao corpo docente quanto ao discente. Em relação aos docentes, a pesquisa qualifica as aulas, atualiza os referenciais pedagógicos adotados em sintonia com as discussões em âmbito nacional e internacional e oferece à sociedade e à própria UTFPR as contribuições específicas destas reflexões. Em relação aos alunos, a pesquisa fomenta a formação do \pdfmarkupcomment{tecnólogo-pesquisador}{Engenheiro(a)-pesquisador(a)?}, isto é, aquele comprometido com o aprimoramento de seus conhecimentos, com o desenvolvimento de novas metodologias e a proposição de soluções para os problemas da área. A pesquisa também complementa os estudos realizados pelos alunos e colabora no desenvolvimento de sua autonomia intelectual.

O incentivo à investigação científica e desenvolvimento tecnológico, diagnosticar e solucionar problemas, é um dos objetivos do curso. Em conformidade com o caráter de Universidade, o curso visa através da articulação, tanto interna quanto externa, de conhecimentos socialmente relevantes que contribuam para formar o quadro dos futuros Engenheiros(as) que venham desempenhar um diferencial no mercado de trabalho, contribuindo dessa forma com profissionais que desempenhem trabalhos de qualidade.

As principais ações de interface do curso com o âmbito científico são por meio do Programa Institucional de Bolsas de Iniciação Científica (PIBIC) e do Programa de Bolsas de Iniciação em Desenvolvimento Tecnológico e Inovação (PIBITI). Os PIBIC/PIBITI têm como meta a inicialização dos discentes em pesquisas científicas e tecnológicas nas diferentes áreas de conhecimento. O programa é apoiado pelo CNPq, Fundação Araucária e UTFPR com a concessão de bolsas, sendo que os alunos também podem participar como voluntários do Programa de Voluntariado em Iniciação Científica e Tecnológica (PVICT). Esses programas objetivam despertar a vocação técnico-científica, incentivar novos talentos potenciais entre os estudantes e contribuir para a formação de recursos humanos para a pesquisa, estimulando pesquisadores produtivos a envolverem alunos de Graduação em atividades técnico-científicas e artístico-culturais. Adicionalmente, o PIBIC/PIBITI/PVICT proporciona aos bolsistas e voluntários, orientados por pesquisador qualificado, a aprendizagem de técnicas e métodos de pesquisa além de estimular o desenvolvimento do ``pensar científico'' e das criatividades decorrentes das condições criadas pelo confronto direto com os questionamentos inerentes à pesquisa. O crescente aumento de projetos homologados e de alunos com Iniciação Científica (IC) ressalta o comprometimento dos docentes do Curso de Engenharia Eletrônica com uma formação sólida e consistente no âmbito científico. \pdfmarkupcomment{No período de 2016-2017}{atualizar}, foram homologados, junto à Pró- Reitoria de Pesquisa e Pós-Graduação, 12 (doze) projetos de pesquisas dos docentes do curso de Engenharia Eletrônica. Adicionalmente, \pdfmarkupcomment{o Curso foi contemplado com 10 (dez) bolsas de PIBIC/PIBITI}{atualizar}, fomentadas pelo CNPq, UTFPR e Fundação Araucária. Os alunos do Curso de Engenharia Eletrônica podem realizar atividades de pesquisa em diferentes áreas do conhecimento, visto que há três Grupos de Pesquisa liderados por docentes do curso cadastrados no Diretório do CNPq e certificados pela Instituição, sendo eles: a) Grupo de Eletrônica Aplicada e Sistemas (GEAS); b) Energia Eolicoelétrica; c) Tecnologia de sistemas em geração, controle e eficiência energética; e d) Grupo de processamento eletrônico de fontes alternativas de energia. Os projetos de pesquisa nos quais os estudantes participam apresentam comprovada qualidade acadêmica, mérito científico e orientação adequada pelos docentes do Curso. A participação dos alunos nesses grupos de pesquisa fornece uma ampliação na formação dos bolsistas/voluntários despertando, assim, a vocação científica e incentivando na preparação para ingressar em Programas de Pós-Graduação.

\nomenclature[A]{PIBIC}{Programa Institucional de Bolsas de Iniciação Científica}

\nomenclature[A]{PIBITI}{Programa de Bolsas de Iniciação em Desenvolvimento Tecnológico e Inovação}

\nomenclature[A]{CNPq}{Conselho Nacional de Desenvolvimento Científico e Tecnológico}

\nomenclature[A]{PVICT}{Programa de Voluntariado em Iniciação Científica e Tecnológica}

\nomenclature[A]{IC}{Iniciação Científica}

\section{ARTICULAÇÃO COM A EXTENSÃO}

A extensão universitária é definida como um processo educativo, cultural e científico que se articula ao ensino e a pesquisa de forma indissociável e institucionalizada, viabilizando a relação transformadora entre Universidade e sociedade. Ela oferece um canal de interlocução entre a sociedade e a Universidade, trazendo as demandas da população e os desafios para o desenvolvimento do país para o centro da pesquisa e deste para a sociedade.

De acordo com o artigo 207 da Constituição Brasileira ``as Universidades gozam de autonomia didático-científica, administrativa e de gestão financeira e patrimonial e obedecerão ao princípio da indissociabilidade entre Ensino, Pesquisa e Extensão''. Sendo assim, ensino, pesquisa e extensão devem ser equivalentes e igualmente tratados por parte das Instituições de Ensino Superior, caso contrário, tais entidades estarão se contrapondo à Constituição.

O Curso de Engenharia Eletrônica do campus Toledo tem como pressuposto básico a integração efetiva da extensão universitária ao ensino e à pesquisa, envolvendo os corpos docente e discente, e sempre levando em consideração o compromisso social da Universidade pública brasileira. Além desta indissociabilidade, outros itens das diretrizes básicas definidas no Plano Nacional de Extensão Universitária são compromissos do Curso de Engenharia Eletrônica do campus Toledo: a busca pela interdisciplinaridade e interprofissionalidade; a identificação das demandas da comunidade e das empresas de modo a trazê-las para dentro da Universidade, visando alavancar pesquisas; a articulação de ações que resultem em impacto na formação dos discentes; o incentivo à troca de saberes entre Universidade e sociedade, através da aplicação de metodologias participativas, visando à democratização do conhecimento e a participação efetiva da comunidade na atuação da Universidade. 

Dentro das atividades do curso, a extensão universitária está inserida em trabalhos de conclusão de curso e nos projetos e programas permanentes de extensão desenvolvidos pelos docentes e demais servidores do curso, sempre visando ao estabelecimento de uma forte relação entre ensino, pesquisa e extensão.

%O Resolução nº 69/2018 – COGEP, Regulamento de Registro e de Inclusão das Atividades de Extensão nos Currículos dos Cursos de Graduação da UTFPR, estabelece normas para a acreditação das atividades curriculares de extensão visando a curricularização da extensão a ser implementada em até três anos a partir da data de aprovação dessa Resolução, portanto, até setembro de 2021. A realização de atividades de extensão é obrigatória para todos os estudantes dos cursos regulares de graduação da UTFPR, em um mínimo de dez por cento da carga horária total do seu curso. 
