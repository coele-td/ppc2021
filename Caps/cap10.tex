\chapter{ESTRUTURA DE APOIO}

\section{ATIVIDADES DE TUTORIA}

As atividades de tutoria são parte fundamental na melhoria do processo de acompanhamento dos discentes, seja no início do curso, com atividades de acolhimento e ambientação, seja durante o curso, nas atividades das disciplinas semipresenciais e não presenciais.

No contexto do curso de Engenharia Eletrônica existem vários grupos estudantis que atuam no processo de ambientação, acolhimento e tutoria dos discentes ingressantes. Estes grupos estudantis são acompanhados por docentes na execução das atividades com os discentes ingressantes. Atualmente, fazem parte destas atividades:

\begin{itemize}
    \item Centro Acadêmico de Engenharia Eletrônica (CAEE);
    \item Equipe Hefestus de Robótica \url{facebook.com/hefestus.utfpr/};
    \item Empresa Júnior Exata.
\end{itemize}
    
A partir da reformulação deste PPC, disciplinas semipresenciais e não presenciais passam a compor a matriz curricular do curso. Desta forma, a Resolução 39/2019  (!!Referenciar) regulamenta a criação e a oferta de unidades curriculares na modalidade semipresencial e na modalidade não presencial, em cursos de Graduação presenciais da UTFPR. Conforme o previsto nessa resolução, será definida uma metodologia de tutoria presencial e não presencial pelos docentes e por monitores discentes preparados para este fim.

\section{TECNOLOGIAS DE INFORMAÇÃO E COMUNICAÇÃO (TIC) NO PROCESSO ENSINO-APRENDIZAGEM}

Os mecanismos de interação são caracterizados como 

\begin{citacao}
    ``o conjunto de estruturas de Tecnologia de Informação e Comunicação (TIC) e os respectivos procedimentos e as formas de utilização que caracterizam a dinâmica da comunicação e da interação entre os sujeitos envolvidos nos processos acadêmicos e de ensino e aprendizagem (que são, basicamente, os docentes, tutores e discentes), no contexto da oferta do curso superior na modalidade à distância'' (BRASIL, 2004) (!! Referenciar - vem do ppc atual).
\end{citacao}


O uso de recursos tecnológicos aplicados à educação e comunicação é importante enquanto podem ilustrar conceitos abstratos complexos e enriquecer o contexto de ensino e aprendizagem. Nesse cenário, complementar as técnicas tradicionais com elementos que facilitem a assimilação dos assuntos abordados e contribuam para que a interação entre alunos e professores se torne mais interessante e produtiva pode representar o diferencial em cursos que exijam alto grau de abstração.

As ferramentas de Tecnologia de Informação e Comunicação (TIC) incluem desde conteúdos digitais bem preparados, que podem ser facilmente disponibilizados, passando pela manutenção de sítios \textit{online}, que se tornam repositórios de informação, chegando a mecanismos mais elaborados de gerenciamento de conteúdo e colaboração.

No Instrumento de Avaliação do Curso a partir de 2012 [77] (!!), e também presente no Indicador 1.16 de 2017 [78] (!!), os mecanismos de interação são caracterizados como o conjunto de estruturas de Tecnologia de Informação e Comunicação (TIC) e os respectivos procedimentos e as formas de utilização que caracterizam a dinâmica da comunicação e da interação entre os sujeitos envolvidos nos processos acadêmicos e de ensino e aprendizagem (que são, basicamente, os docentes, tutores e discentes), no contexto da oferta do curso superior na modalidade a distância.

%[77] Universidade Tecnológica Federal do Paraná, “Deliberação COUNI nº 13/2009 - Atualização Regulamento CPA,” Curitiba, 2009.
%[78] Ministério da Educação, “Instrumento de Avaliação de Cursos de Graduação - 2012,” Brasília, 2012.


A instituição disponibiliza alguns ambientes e artefatos de comunicação para mediarem atividades didáticas nas modalidades presencial e não presencial:

\begin{itemize}
    \item Página pessoal docente;
    \item Moodle institucional;
    \item Plataforma GSuite for Education;
    \item Serviço Mconf em parceria com a Rede Nacional de Ensino e Pesquisa (RNP);
    \item Base Digital de dados;
    \item Repositórios institucionais;
    \item Office 365.
\end{itemize}
    
Diante disso, o processo de ensino aprendizagem é intensificado com o uso das TIC e demais artefatos tecnológicos, por meio de atividades de comunicação, colaboração e compartilhamento, propiciando a construção e a produção de conhecimentos e o desenvolvimento de habilidades pessoais e interpessoais do corpo discente, e fomentando novas práticas do docente.

Os recursos tecnológicos disponíveis no campus são mediados pela Coordenação de Gestão de Tecnologia da Informação (COGETI), sendo responsável pelo Moodle institucional. Adicionalmente, dá suporte à infraestrutura de redes, manutenção de computadores e instalação de softwares. Além disso, o sistema de bibliotecas disponibiliza ampla base digital de dados e repositórios institucionais para produção acadêmica em geral.

\section{AMBIENTES DE APRENDIZAGEM (PRESENCIAL/HÍBRIDO/EAD)}

Com relação à infraestrutura dos ambientes de ensino e aprendizagem, atualmente o Campus dispõe de:

\begin{itemize}
    \item 18 salas de aula com capacidade para 50 alunos cada, sendo que estas são equipadas com projetor multimídia, ventiladores e quadro branco e 2 salas com capacidade de 24 alunos com os mesmos equipamentos listados anteriormente;
    \item Uma sala de atendimento de monitoria;
    \item Uma sala de estudo 24 horas;
    \item Sete laboratórios de informática para aulas teóricas ou práticas que necessitem de softwares;
    \item Seis laboratórios de exclusivos para os alunos do curso de Engenharia Eletrônica;
    \item Dois laboratórios de aulas práticas de Física;
    \item Sete laboratórios para aulas práticas de química com capacidade para 24 alunos cada;
    \item Uma biblioteca, com 5 salas de estudo, mesas de estudos individuais, livros da bibliográfica básica e complementar (entre outros), revistas, periódicos, computadores com acesso a rede e equipado com os softwares utilizados em nos laboratórios de informática.
\end{itemize}

Já em termos de ambientes virtuais de aprendizagem, estes devem proporcionar a discente e docente recursos que facilitem a execução de atividades síncronas e assíncronas, bem como meios para interação e devolutiva ao discente. Atualmente, a UTFPR disponibiliza ao menos dois ambientes virtuais de ensino aprendizagem (AVEA): Moodle e Gsuit for Education.
O ambiente Moodle é uma ferramenta já amplamente empregada em disciplinas presenciais, semipresenciais e não presenciais. Trata-se de um sistema de gerenciamento de aprendizagem open-source e gratuíto. Pode ser utilizado para diversos fins educacionais, seja na educação a distância, organização de materiais educacionais e auxílio a metodologias ativas.

O ambiente Gsuit for Education é uma plataforma educativa colaborativa, que une diversos recursos disponíveis do Google. Nesta plataforma é possível utilizar editor de texto, apresentação de slides, planilhas, agenda, e drive, todos de forma colaborativa. Além disso, o Google Classroom auxilia no gerenciamento de uma sala virtual e encontros síncronos podem ocorrer com a ferramenta Google Meet.

\section{MATERIAL DIDÁTICO}

A respeito de materiais didáticos, há uma ampla gama de materiais que podem ser aplicados às disciplinas presenciais, semipresenciais e/ou não presenciais. Gravação de aulas, tutoriais, apostilas, guias práticos são recursos que cada professor utiliza/desenvolve, de acordo com a necessidade, para sua disciplina. Estes materiais são mantidos em repositório próprio do professor, tal como página pessoal, Moodle, Google Drive, Youtube e demais plataformas disponíveis.

Além disso, o sistema de bibliotecas disponibiliza ampla base digital de dados e repositórios institucionais para uso didático em geral.

\section{INFRAESTRUTURA DE APOIO ACADÊMICO}

A estrutura de apoio é entendida por apoio pedagógico e infraestrutura física de apoio acadêmico, de ensino e de aprendizagem.

Em termos de estrutura de apoio pedagógico, a UTFPR conta com o Departamento de Educação (DEPED) voltado à consolidação e melhoria do processo de ensino aprendizagem, conforme estabelece o Regimento Geral da UTFPR.

O DEPED é composto por:

\begin{itemize}
    \item Núcleo de Ensino (NUENS) voltado à gestão pedagógica e o atendimento direto aos docentes;
    \item Núcleo de Acompanhamento Psicopedagógico e Assistência Estudantil (NUAPE) voltado ao atendimento coletivo e individualizado dos discentes.
\end{itemize}

Além disso, a UTFPR tem começado a implementar em seus campi o Núcleo de Acessibilidade e Inclusão (NAI). Esta estrutura busca prover recursos e serviços, de acordo com as necessidades individuais dos estudantes com deficiência (PcD), transtorno do espectro autista e altas habilidades ou superdotação. O intuito é eliminar fatores que restringem ou impedem a participação plena e o desenvolvimento acadêmico e social, em condição de igualdade com as demais pessoas.

Em termos de infraestrutura física de apoio acadêmico, de ensino e de aprendizagem, a UTFPR conta com a Secretaria de Gestão Acadêmica (SEGEA) e Coordenação de Gestão de Tecnologia da Informação (COGETI).

A relação entre docente e a infraestrutura de apoio pode ocorrer de forma direta, de acordo com demandas pontuais ou em momentos de capacitação e orientação aos docentes, como também de forma indireta por meio da coordenação do curso.


\section{INSTALAÇÕES GERAIS E ESPECÍFICAS}

As instalações do Curso de Engenharia Eletrônica fazem parte do Campus Toledo e funcionam atualmente sito à rua Cristo Rei, 19 – Vila Becker – CEP: 85902-490 – Toledo – PR.

As atividades de ensino no Campus Toledo são realizadas nos blocos A, C, E e nos laboratórios anexos à quadra de esportes. As instalações do curso de Engenharia Eletrônica do Campus de Toledo se concentram no Bloco A que possui como infraestrutura 3381 m\textsuperscript{2} dispostas em quatro pavimentos constituídos de laboratórios, salas de aula e áreas administrativas descritos no Quadro 21. Eventualmente, outros ambientes dos blocos C e E são empregados nas atividades de ensino.






