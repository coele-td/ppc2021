\chapter{ESTRUTURA DE APOIO}

\section{ATIVIDADES DE TUTORIA}

As atividades de tutoria são parte fundamental na melhoria do processo de acompanhamento dos discentes, seja no início do curso, com atividades de acolhimento e ambientação, seja durante o curso, nas atividades das disciplinas semipresenciais e não presenciais.

No contexto do curso de Engenharia Eletrônica existem vários grupos estudantis que atuam no processo de ambientação, acolhimento e tutoria dos discentes ingressantes. Estes grupos estudantis são acompanhados por docentes na execução das atividades com os discentes ingressantes. Atualmente, fazem parte destas atividades:

\begin{itemize}
    \item Centro Acadêmico de Engenharia Eletrônica (CAEE);
    \item Equipe Hefestus de Robótica \url{facebook.com/hefestus.utfpr/};
    \item Empresa Júnior Exata.
\end{itemize}
    
A partir da reformulação deste PPC, disciplinas semipresenciais e não presenciais passam a compor a matriz curricular do curso. Desta forma, a Resolução 39/2019  (!!Referenciar) regulamenta a criação e a oferta de unidades curriculares na modalidade semipresencial e na modalidade não presencial, em cursos de Graduação presenciais da UTFPR. Conforme o previsto nessa resolução, será definida uma metodologia de tutoria presencial e não presencial pelos docentes e por monitores discentes preparados para este fim.

\section{TECNOLOGIAS DE INFORMAÇÃO E COMUNICAÇÃO (TIC) NO PROCESSO ENSINO-APRENDIZAGEM}

Os mecanismos de interação são caracterizados como 

\begin{citacao}
    ``o conjunto de estruturas de Tecnologia de Informação e Comunicação (TIC) e os respectivos procedimentos e as formas de utilização que caracterizam a dinâmica da comunicação e da interação entre os sujeitos envolvidos nos processos acadêmicos e de ensino e aprendizagem (que são, basicamente, os docentes, tutores e discentes), no contexto da oferta do curso superior na modalidade à distância'' (BRASIL, 2004) (!! Referenciar - vem do ppc atual).
\end{citacao}


O uso de recursos tecnológicos aplicados à educação e comunicação é importante enquanto podem ilustrar conceitos abstratos complexos e enriquecer o contexto de ensino e aprendizagem. Nesse cenário, complementar as técnicas tradicionais com elementos que facilitem a assimilação dos assuntos abordados e contribuam para que a interação entre alunos e professores se torne mais interessante e produtiva pode representar o diferencial em cursos que exijam alto grau de abstração.

As ferramentas de Tecnologia de Informação e Comunicação (TIC) incluem desde conteúdos digitais bem preparados, que podem ser facilmente disponibilizados, passando pela manutenção de sítios \textit{online}, que se tornam repositórios de informação, chegando a mecanismos mais elaborados de gerenciamento de conteúdo e colaboração.

No Instrumento de Avaliação do Curso a partir de 2012 [77] (!!), e também presente no Indicador 1.16 de 2017 [78] (!!), os mecanismos de interação são caracterizados como o conjunto de estruturas de Tecnologia de Informação e Comunicação (TIC) e os respectivos procedimentos e as formas de utilização que caracterizam a dinâmica da comunicação e da interação entre os sujeitos envolvidos nos processos acadêmicos e de ensino e aprendizagem (que são, basicamente, os docentes, tutores e discentes), no contexto da oferta do curso superior na modalidade a distância.
A instituição disponibiliza alguns ambientes e artefatos de comunicação para mediarem atividades didáticas nas modalidades presencial e não presencial:

\begin{itemize}
    \item Página pessoal docente;
    \item Moodle institucional;
    \item Plataforma GSuite for Education;
    \item Serviço Mconf em parceria com a Rede Nacional de Ensino e Pesquisa (RNP);
    \item Base Digital de dados;
    \item Repositórios institucionais;
\end{itemize}
    
Diante disso, o processo de ensino aprendizagem é intensificado com o uso das TIC e demais artefatos tecnológicos, por meio de atividades de comunicação, colaboração e compartilhamento, propiciando a construção e a produção de conhecimentos e o desenvolvimento de habilidades pessoais e interpessoais do corpo discente, e fomentando novas práticas do docente.

Os recursos tecnológicos disponíveis no campus são mediados pela Coordenação de Tecnologia na Educação (COTED) e Coordenação de Gestão de Tecnologia da Informação (COGETI). A COTED dá suporte ao Ambiente Virtual de Ensino e Aprendizagem (AVEA) e a produção de recursos educacionais digitais. Enquanto, a COGETI dá suporte à infraestrutura de redes, manutenção de computadores e instalação de softwares. Além disso, o sistema de bibliotecas disponibiliza ampla base digital de dados e repositórios institucionais para produção acadêmica em geral.

