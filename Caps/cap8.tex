\chapter{AVALIAÇÃO INSTITUCIONAL}

A avaliação institucional é um processo planejado e normatizado na UTFPR. A partir dos indicadores obtidos pelas avaliações, a gestão do curso define encaminhamentos para orientar a melhoria contínua da qualidade, eficiência, eficácia e publicidade, entendidas como princípios que agregam valor às atividades desenvolvidas pela Instituição \cite{pdiutfpr}. 

O processo de avaliação institucional é composto por diversos instrumentos, tanto externos quanto internos, cujo acompanhamento, análise e \textit{feedback} são realizados pela Comissão própria de Avaliação (CPA).

\nomenclature[A]{CPA}{Comissão Própria de Avaliação}

\section{Comissão própria de avaliação}

A CPA da UTFPR tem por finalidade o planejamento, o desenvolvimento, a coordenação e a supervisão da política de avaliação institucional.

A CPA iniciou suas atividades em dezembro de 2004 \cite{couni8} e, com a transformação de CEFET-PR em UTFPR, o seu regulamento foi atualizado pela Deliberação COUNI N\textordmasculine{} 13/2009 \cite{couni13}. A página da CPA na internet está disponível no endereço: \url{http://portal.utfpr.edu.br/comissoes/permanentes/cpa}.

\section{POLÍTICA INSTITUCIONAL DE AVALIAÇÃO (INTERNA)}

No âmbito da avaliação interna, a UTFPR vem desenvolvendo e aprimorando instrumentos de acompanhamento e de avaliação, com destaque para:

\begin{itemize}
    \item Levantamento do perfil socioeconômico e educacional dos estudantes; 
    \item Avaliação do desempenho dos servidores da UTFPR (docentes e técnico administrativos); do docente pelo discente; do servidor em função de chefia, pela equipe de trabalho; e do desempenho coletivo de setores da Instituição, sob a perspectiva dos usuários;
    \item Pesquisa de clima organizacional; de satisfação do cliente externo.
\end{itemize}

\section{AVALIAÇÃO EXTERNA}

A avaliação institucional externa, de cursos e o ENADE são executados pelo INEP (Instituto Nacional de Estudos e Pesquisas Educacionais Anísio Teixeira), vinculado ao MEC. O conhecimento dos resultados da avaliação, associado às mudanças e aos desafios que vêm se apresentando para a sociedade como um todo, possibilita que UTFPR estabeleça novos patamares institucionais, no sentido acadêmico e como indutora do desenvolvimento sustentável e de relevância social no seu entorno.

\section{Avaliação do corpo docente}

A UTFPR trabalha com uma avaliação semestral dos docentes feita pelos discentes. Esta avaliação é um importante instrumento de acompanhamento da qualidade de ensino oferecido, proporcionando aos alunos uma participação efetiva na busca pela excelência do ensino.

O instrumento busca evitar o caráter punitivo, constituindo uma avaliação construtiva, e oferece aos docentes um retorno dos alunos sobre sua atuação. As avaliações são realizadas através de formulários eletrônicos, disponibilizados no sistema acadêmico, e podem ser acessados conforme a disponibilidade do aluno no período de avaliações. Os resultados não apresentam nenhum tipo de identificação pessoal dos alunos, e permanecem no banco de dados, e são processados pela Diretoria de Gestão da Tecnologia da Informação (DIRGTI), sendo divulgados aos Departamentos Acadêmicos e Coordenações de Curso somente após o término do semestre letivo, para que os alunos não se sintam inibidos em realizar a avaliação. Após o acesso aos resultados serem liberados aos docentes, a coordenação de curso busca dialogar com os mesmos, identificando os pontos fortes e fraquezas, de modo a colaborar com o processo.

\nomenclature[A]{DIRGTI}{Diretoria de Gestão da Tecnologia da Informação}

O campus conta com duas comissões específicas para acompanhar o processo de avaliação do docente pelo discente, a comissão de aplicação e a comissão pedagógica. A comissão de aplicação é responsável pela aplicação do processo avaliativo, acompanhando os índices de participação dos alunos, detectando os motivos causadores de baixos índices de participação e incentivando a participação. A comissão pedagógica, em conjunto com o Coordenador de Curso realiza a devolutiva dos resultados e propõe atividades para reparar pontos frágeis e aprimorar a prática docente.

O docente também tem seu desempenho avaliado pela chefia, através da avaliação desenvolvida pela coordenação de recursos humanos, por meio do Sistema de Avaliação Institucional (SIAVI). Este processo de avaliação serve como parâmetro para avaliar a instituição, comportamentos e chefias, estando intimamente relacionado com as atividades de planejamento e gestão de resultados. A avaliação de desempenho fornece subsídios à área de recursos humanos, considerando a capacitação e carreira dos servidores.

\nomenclature[A]{SIAVI}{Sistema de Avaliação Institucional}

Além dos instrumentos institucionais que realizam a avaliação de desempenho dos docentes, por sugestão da coordenação de Curso e do Departamento de Educação, os professores são aconselhados a realizar uma avaliação de sua disciplina e de seu desempenho em sala de aula ao final de cada disciplina, buscando a melhoria dos processos de ensino e aprendizagem. Visando a complementar os instrumentos já utilizados para a avaliação do docente, o curso desenvolveu um instrumento próprio de auto avaliação, que contempla também a atuação do docente.

Como a avaliação do docente pelos alunos já é contemplada na avaliação institucional, o instrumento de autoavaliação do curso apresenta um enfoque maior na autoavaliação do docente acerca de sua atuação nos componentes curriculares. Dessa forma, é realizada a autoavaliação do docente, com o instrumento de autoavaliação do curso, a avaliação, através da avaliação do docente pelo discente, e a heteroavaliação, através da avaliação de desempenho do servidor.

\section{Avaliação do curso}

Os mecanismos de avaliação permanente da efetividade do processo de ensino e aprendizagem visam compatibilizar a oferta de vagas e o modelo do curso com a demanda do mercado de trabalho. O principal mecanismo utilizado para a avaliação do curso é o Sistema Nacional de Avaliação da Educação Superior (SINAES) que, através do Decreto No 5.773, de 9 de maio de 2006, dispõe sobre o exercício das funções de regulação, supervisão e avaliação de instituições de educação superior e cursos superiores de graduação e sequenciais no sistema federal de ensino. Esta avaliação tem como componentes os seguintes itens:

\nomenclature[A]{SINAES}{Sistema Nacional de Avaliação da Educação Superior}

\begin{itemize}
    \item Autoavaliação, conduzida pelas CPAs;
    \item Avaliação externa, realizada por comissões externas designadas pelo INEP;
    \item Avaliação dos cursos de graduação (ACG);
    \item ENADE – Exame Nacional de Avaliação de Desenvolvimento dos estudantes.
\end{itemize}
     
\nomenclature[A]{INEP}{Instituto Nacional de Estudos e Pesquisas Educacionais Anísio Teixeira}
\nomenclature[A]{ACG}{Avaliação dos cursos de graduação}
\nomenclature[A]{ENADE}{Exame Nacional de Avaliação de Desenvolvimento dos Estudantes}

Visando ao aperfeiçoamento contínuo do curso, o NDE faz uma avaliação semestral das atividades realizadas no período, sempre discutindo formas de melhorar a atuação da coordenação e dos docentes. 

\section{Avaliação institucional}

A avaliação institucional é de responsabilidade da CPA composta por membros da comunidade acadêmica e da sociedade civil organizada, formando um colegiado, para planejar e executar a avaliação institucional no âmbito do Sistema Nacional de Avaliação do Ensino Superior (SINAES), estabelecido pela Lei 10.861, de 14/04/2004 \cite{Lei:10861:2004}.

As Instituições de Ensino Superior (IES) são avaliadas em três momentos:

\begin{enumerate}
    \item Avaliação institucional (auto avaliação e avaliação externa);
    \item Avaliação dos cursos;
    \item Exame Nacional de Desempenho do Estudante (ENADE).
\end{enumerate}

A avaliação institucional externa, de cursos e o ENADE são executados pelo INEP (Instituto Nacional de Estudos e Pesquisas Educacionais Anísio Teixeira), vinculado ao Ministério da Educação (MEC). É responsabilidade da CPA executar a autoavaliação institucional. Nesse contexto, a avaliação dos servidores é composta pela avaliação individual do servidor (realizada pela chefia imediata do servidor), avaliação do docente pelo discente, avaliação dos setores pelos usuários, e avaliação das chefias pelos subordinados. A avaliação individual do servidor é realizada anualmente pela chefia imediata do servidor, compondo parte de sua nota na avaliação de desempenho. Essa avaliação é complementada pela avaliação do docente pelo discente, no caso dos professores, e pela avaliação do setor pelo usuário, no caso dos servidores técnico administrativos. A avaliação de clima organizacional também é realizada pela instituição, com o objetivo de identificar as fragilidades e fortalezas institucionais. Todos os instrumentos utilizados nas avaliações são informatizados.

\section{ACOMPANHAMENTO DO EGRESSO}

Dentre os vários indicadores de qualidade de uma Instituição de Ensino Superior destacam se os resultados de investigações empíricas sobre o acompanhamento da vida profissional e educacional de seus egressos. Egresso é todo estudante que concluiu seus estudos no ensino de graduação ou pós-graduação, e como tal pode continuar com vínculos não só afetivos, mas que também participem de atividades que a instituição organiza e desenvolve na área do ensino, pesquisa e extensão, em graus e níveis distintos.

O acompanhamento do egresso é um elemento importante para avaliação e revisão do curso, especialmente, no que se refere a relação entre currículo e mundo do trabalho. O setor responsável pelo acompanhamento dos egressos na UTFPR, atualmente, é a Pró-Reitoria de Relações Empresariais e Comunitárias (PROREC). O acompanhamento de egressos realizado pela UTFPR tem como principais objetivos:

\begin{itemize}
    \item Propiciar à UTFPR o cadastramento dos principais empregadores dos nossos egressos, bem como um cadastro atualizado dos nossos ex-alunos;
    \item Desenvolver meios para a avaliação e adequação dos currículos dos cursos, através da realimentação por parte da sociedade e especialmente dos ex-alunos;
    \item Criar condições para a avaliação de desempenho dos egressos em seus postos de trabalho;
    \item Criar indicadores confiáveis para a avaliação contínua dos métodos e técnicas didáticas e conteúdos empregados pela instituição no processo ensino aprendizagem;
    \item Dispor de informações atualizadas dos nossos ex-alunos, objetivando informá-los sobre eventos, cursos, atividades e oportunidades oferecidas pela Instituição;
    \item Disponibilizar aos nossos formandos oportunidades de emprego, disponibilizadas à DIREC por parte das empresas e agências de recrutamento e seleção de pessoal.
\end{itemize}    

Os egressos do curso de Engenharia Eletrônica sempre são convidados a participar de mesas redondas, semanas acadêmicas para apresentar suas experiências profissionais. Os egressos atuam nas mais diversas áreas como indústria, laboratórios de análises, gestão da qualidade, pesquisa e pós-graduação.
