\chapter{ARTICULAÇÃO COM OS VALORES, PRINCÍPIOS E POLÍTICAS DE ENSINO DA UTFPR}

Este Capítulo é uma complementação do \autoref{chap:politicas}, apresentado como é a estruturação das políticas de ensino da UTFPR no âmbito do Curso de Engenharia Eletrônica, tendo como base a matriz curricular apresentada no \autoref{chap:matriz}.

\section{DESENVOLVIMENTO DA ARTICULAÇÃO ENTRE A TEORIA E A PRÁTICA}

O curso de Engenharia Eletrônica define quatro instrumentos para desenvolver a articulação entre a teoria e a prática: (i) atividades práticas em unidades curriculares, (ii) unidades curriculares integradoras e interdisciplinares, (iii) estágio curricular obrigatório e (iv) Trabalho de Conclusão de Curso (TCC).

Neste sentido, a matriz curricular apresenta 53 unidades curriculares (mais 225 h de unidades curriculares optativas do ciclo de humanidades), das quais 34 contemplam carga horária prática, totalizando \the\value{horasAP} horas práticas no curso de Engenharia Eletrônica. Informações mais detalhadas sobre esta distribuição podem ser observadas no \autoref{qua:matriz} na \autoref{sec:matriz}, Matriz Curricular.

As unidades curriculares integradoras e interdisciplinares, que serão listadas na \autoref{sec:comp}, preveem a realização de experiências práticas de implementação de projetos, complementando a articulação entre teoria e prática, e inserindo a perspectiva de interdisciplinaridade no curso.

O estágio curricular obrigatório contabiliza 400 h de atividades, podendo ser iniciado a partir do 7\textordmasculine{} período. Neste contexto, o estudante é capaz de colocar em prática todo o ensinamento recebido durante seus anos de estudo no curso, sendo acompanhado por um professor orientador e um supervisor responsável pelo estágio na empresa que o oferece. Para possibilitar um melhor aproveitamento do discente em relação ao Estágio, este PPC foi desenvolvido prevendo a redução de carga horária no nono e décimo período. Desta forma, o discente tem condições de buscar oportunidades de estágio em outras regiões, dentro e fora do país.

Assim como o estágio obrigatório, o TCC é capaz de colocar em prática todo o ensinamento recebido pelo discente durante o curso, sendo acompanhado por um(a) docente orientador(a). Os(as) orientadores, em consonância com a atuação do docente responsável pelas atividades de TCC, buscam sempre a realização de projetos práticos, como é caso do desenvolvimento de processos, produtos ou protótipos.

\section{DESENVOLVIMENTO DAS COMPETÊNCIAS PROFISSIONAIS}
\label{sec:comp}

Assim como citado na \autoref{sec:perf}, neste PPC são propostas quatro Competências específicas para o egresso do curso: Eletrônica, Científica, Empreendedora e Eletrotécnica.

\subsection{A competência de Eletrônica}

A competência Eletrônica é a mais importante e complexa do curso. Para ser atingida é necessária que o discente seja aprovado em quatro unidades curriculares integradoras e interdisciplinares: Medidas e Sensores, Sistemas Embarcados, Processamento Digital de Sinais e Controle e Supervisório. Essas são disciplinas práticas que representam as principais áreas que um Engenheiro Eletrônico pode atuar. Todas possuem a carga horária de atividades práticas maior ou igual a 30 h, propiciando a execução de atividades ou projetos integradores, onde os docentes podem avaliar se o discente atingiu a Competência.

Ao conquistar essa competência, o discente será capaz de \textbf{conceber e/ou intervir em sistemas eletrônicos com autonomia, integrando circuitos analógicos, computação embarcada, controle de sistemas e processamento digital de informações e considerando uma documentação clara e concisa}.


% \begin{itemize}
% 	\item \textbf{Eletrônica}: conceber e/ou intervir em sistemas eletrônicos com autonomia, integrando circuitos analógicos, computação embarcada, controle de sistemas e 	processamento digital de informações e considerando uma documentação clara e concisa;
% 	\item \textbf{Científica}: produzir investigação científica integrando modelos de fenômenos naturais, conhecimentos técnico-científicos, escrita e metodologia científica com honestidade intelectual e senso crítico;
% 	\item \textbf{Empreendedora}: Analisar ou propor negócios, com responsabilidade compartilhada e atitudes empreendedora e cooperativa, por meio da articulação de informações técnicas e conceituais e avaliação do micro e macroambiente;
% 	\item \textbf{Eletrotécnica}: Conceber e/ou intervir em sistemas elétricos com autonomia, integrando instalações elétricas , máquinas elétricas e sistemas de potência, considerando uma documentação clara e concisa e segurança elétrica.
% \end{itemize}

\section{DESENVOLVIMENTO DA FLEXIBILIDADE CURRICULAR}

\section{DESENVOLVIMENTO DA MOBILIDADE ACADÊMICA}

\section{DESENVOLVIMENTO DA INTERNACIONALIZAÇÃO}

\section{DESENVOLVIMENTO DA ARTICULAÇÃO COM A PESQUISA E PÓS GRADUAÇÃO}

\section{DESENVOLVIMENTO DA EXTENSÃO}

\subsection{Projetos e unidades curriculares extensionistas}