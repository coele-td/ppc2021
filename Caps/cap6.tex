\chapter{ARTICULAÇÃO COM OS VALORES, PRINCÍPIOS E POLÍTICAS DE ENSINO DA UTFPR}

Este Capítulo é uma complementação do \autoref{chap:politicas}, apresentado como é a estruturação das políticas de ensino da UTFPR no âmbito do Curso de Engenharia Eletrônica, tendo como base a matriz curricular apresentada no \autoref{chap:matriz}.

\section{DESENVOLVIMENTO DA ARTICULAÇÃO ENTRE A TEORIA E A PRÁTICA}

Como descrito na \autoref{sec:artc}, ARTICULAÇÃO ENTRE A TEORIA E A PRÁTICA E INTERDISCIPLINARIDADE, o curso de Engenharia Eletrônica define três instrumentos para realizar este processo: (i) atividades práticas em unidades curriculares, (ii) unidades curriculares integradoras e interdisciplinares e (iii) estágio curricular obrigatório.

Neste sentido, a matriz curricular apresenta 58 unidades curriculares, das quais 34 contemplam carga horária prática, totalizando \the\value{horasAP} horas práticas no curso de Engenharia Eletrônica. Informações mais detalhadas sobre esta distribuição podem ser observadas no \autoref{qua:matriz} na \autoref{sec:matriz}, Matriz Curricular.



O estágio curricular obrigatório contabiliza 400 h de atividades, podendo ser iniciado a partir do 7\textordmasculine período. Neste contexto, o estudante é capaz de colocar em prática todo o ensinamento recebido durante seus anos de estudo no curso, sendo acompanhado por um professor orientador e um supervisor responsável pelo estágio na empresa que o oferece. Para possibilitar um melhor aproveitamento do discente em relação ao Estágio, este PPC foi desenvolvido prevendo a redução de carga horária no nono e décimo período. Desta forma, o discente tem condições de buscar oportunidades de estágio em outras regiões, dentro e fora do país.

\section{DESENVOLVIMENTO DAS COMPETÊNCIAS PROFISSIONAIS}
\label{sec:comp}