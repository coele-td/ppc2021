\chapter{POLÍTICA INSTITUCIONAL DE DESENVOLVIMENTO PROFISSIONAL DOCENTE}

Como instituição comprometida com a formação inicial e continuada, a UTFPR dispõe de um Programa de Desenvolvimento Profissional Docente da UTFPR, aprovado pela Resolução COGEP N\textordmasculine{} 32/2019 \cite{cogep32}, com finalidade do aperfeiçoamento da prática docente, possibilitando a busca de alternativas às dificuldades que envolvem os processos de ensino e aprendizagem na Instituição.

Conceitua-se no meio acadêmico, em especial o universitário, que a palavra formação é ``entendida como um processo que tende a desenvolver no adulto certas capacidades mais específicas com vistas a desempenhar um papel particular que implica em um conjunto definido de técnicas e tarefas'' \cite[p. 25]{vaillant2012ensinando}. Esse processo de formação é um fenômeno complexo e diverso que se vincula com a capacidade dos sujeitos envolvidos bem como com a sua vontade. Significa dizer que é o indivíduo, a pessoa o responsável pela ativação e desenvolvimento dos processos formativos. No entanto, é também por meio da formação mútua que os sujeitos podem encontrar contextos de aprendizagem que favoreçam à busca de metas de aperfeiçoamento pessoal e profissional.

Nesse sentido, \citeauthoronline{vaillant2012ensinando} \cite{vaillant2012ensinando} elucidam alguns conceitos necessários ao contexto da formação, tais como: autoformação, heteroformação e interformação. Para esses autores, a autoformação é uma formação em que o sujeito participa de forma independente e possui o controle dos seus objetivos, dos seus processos, dos seus instrumentos e dos resultados da própria formação. Já a heteroformação se organiza e se desenvolve ``de fora'', por especialistas, sem que seja comprometida a personalidade do sujeito participante e finalmente a interformação é aquela que se produz em contextos de trabalho em equipe.

O NDE acredita que de tudo o que foi conceituado até o momento, para a UTFPR, a formação é um processo individual e social. E nesse sentido, para além da formação, há que se considerar que os profissionais da educação estão envoltos nos processos de ensino e aprendizagem em seus diferentes contextos e principalmente, lembrando que estamos formando adultos. Assim, é necessário, a cada semestre repensar os contextos de formação e as conexões que os mesmos estabelecem com a prática profissional. Inseridos no contexto universitário, há a necessidade de repensar os processos que abarcam o fazer docente e nele situa-se o processo de ensino e aprendizagem. O processo de ensino e aprendizagem reveste-se de nuances que envolvem o ato de planejar, executar, avaliar, num ciclo que não se encerra: é um processo dialógico e dialético, portanto sempre inacabado.

Nesse processo estão em jogo negociações, aprendizagens, ensinos, trocas de experiências que enriquecem nosso fazer pedagógico e possibilitam nossa autoformação, heteroformação e interformação. Por se tratar de um processo contínuo, a cada etapa novas necessidades vão surgindo, novas exigências gestoras, educativas, sociais, tecnológicas e culturais vão se apresentando e é necessário rediscuti-las, confrontá-las, analisá-las e melhorá-las.

Em seu Plano de Desenvolvimento Institucional \cite{pdiutfpr} a UTFPR em sua estrutura organizacional conta com o Departamento de Educação vinculado à PROGRAD que tem como ações diretamente ligadas ao processo de ensino e aprendizagem e de formação continuada as seguintes:

\begin{itemize}
    \item Desenvolver uma política institucional para os programas de educação continuada para os coordenadores e professores de cursos da UTFPR;
    \item Em cada Campus, o Departamento de Educação visa implementar ações para aplicação das políticas visando melhorias para o desenvolvimento do processo ensino-aprendizagem.
\end{itemize}

Nesse sentido, o período de planejar e de formação é fundamental para refletir, discutir, acordar, discordar, mas acima de tudo refletir sobre a experiência vivida, pois segundo \citeauthoronline{vaillant2012ensinando} \cite[p. 41]{vaillant2012ensinando} ``a análise da prática observada ou experimentada, à luz das crenças e conhecimentos próprios, permite pôr em questão as próprias ideias e avançar em direção a uma maior autoconsciência do conhecimento profissional''.

Assim, a Diretoria de Graduação e Educação Profissional por meio do seu Departamento de Educação propõe continuamente no início de cada semestre letivo os Projetos de Planejamento Educacional para o Campus de Toledo da UTFPR. Dessa forma, envolve-se todos os seus profissionais da educação, conforme objetivos e cronogramas executados após consulta efetuada com seus docentes e coordenadores de curso nos momentos de colegiado e individualmente, sob a ótica das avaliações realizadas no primeiro e segundo semestre de cada ano letivo dos docentes pelos discentes, nos resultados apontados pelos relatórios de gestão e autoavaliação e pelas metas que o DEPED almeja alcançar nos processos de autoformação, heteroformação e interformação com todos os profissionais da educação.

O período de Planejamento de Ensino e Capacitação Docente é desenvolvido por palestras, minicursos, reuniões e planejamento de ensino. As palestras têm como meta suscitar debates em torno do aluno que temos hoje na Universidade: conectado ao mundo virtual e digital, com forte apelo midiático, com parca formação científica básica, pertencente ao mundo contemporâneo, ao qual o professor precisa ficar ``antenado'' sob pena de ser ultrapassado em seus métodos e técnicas de trabalho e diálogo em sala de aula, as temáticas da inclusão e a própria formação do professor e do profissional, bem como aprofundar temáticas relacionadas a metodologias de ensino.

Os minicursos são proposições oriundas da necessidade levantada pelos docentes e técnicos administrativos que vislumbram esse período formativo como ideal para ampliar suas competências e habilidades laborais e tecnológicas, bem como advém das demandas propostas pela Comissão Própria de Avaliação em relação à avaliação dos cursos.

As reuniões são os espaços de discussão e proposição dos diferentes grupos de trabalho, que tem a sua frente professores/as como líderes de diferentes comissões que necessitam planejar, fazer devolutivas de trabalhos realizados, bem como dar prosseguimento a trabalhos iniciados em cada ano letivo. Também é o espaço em que a equipe gestora do Campus pode repassar informações, planejar ações coletivas e apresentar as normativas que se fizerem necessárias para a continuidade dos trabalhos que serão efetivados no primeiro e segundo semestre de cada ano letivo.

Além das ações propostas nas Semanas de Capacitação e Planejamento anualmente, o Campus Toledo da UTFPR tem uma ``Proposta de Formação Pedagógica Continuada para os Docentes''. Essa proposta é uma iniciativa do Departamento de Educação (DEPED) e que por sua vez responde à Diretoria de Graduação e Educação Profissional (DIRGRAD) (UTFPR, 2009).