\chapter*{Apresentação}

Este documento, contém o Projeto Pedagógico do Curso de \textbf{Engenharia Eletrônica}  do câmpus Toledo, da Universidade Tecnológica Federal do Paraná (UTFPR) \cite{abertura, reconhecimento, renova1, renova2, renova3}. É o resultado do trabalho coletivo entre Núcleo Docente Estruturante (NDE), Colegiado de Curso, professores e da Diretoria de Graduação (DIRGRAD-TD), considerando as legislações e normas nacionais, as institucionais, as especificidades e demandas locais, conforme estabelecido no art.13\textordmasculine{} da Lei de Diretrizes e Bases da Educação Nacional \cite{Lei:9394:1996}.

Nele estão contidas as ações educativas e as características necessárias ao curso para cumprir seus propósitos e suas intencionalidades. Deve ser conhecido e utilizado como importante norteador de suas ações, pelos profissionais a ele vinculado direta ou indiretamente.

Ao expressar a organização do curso em seu todo, o documento demonstra a importância do papel social da universidade pública, das ações comprometidas com o ensino de qualidade e excelência, para o enfrentamento de novos desafios profissionais e humanos, atribuindo centralidade à flexibilidade curricular, ao empreendedorismo e à inovação.

Consequentemente, este projeto está de acordo com as Diretrizes Curriculares dos Cursos de Graduação Regulares da Universidade Tecnológica Federal do Paraná \cite{cogep90}, com as Diretrizes Curriculares Nacionais do Curso de Graduação em Engenharia \cite{dcneng} e com as Diretrizes Institucionais Específicas para a habilitação profissional pretendida.

O compromisso com a formação de cidadãos capazes de propor soluções tecnicamente acertadas e considerar os problemas diversos em sua totalidade e múltiplas dimensões está presente desde o perfil do egresso e objetivos do curso, perpassando pelos valores e princípios institucionais, políticas de ensino, até o desenvolvimento de projetos e disciplinas extensionistas.

Neste contexto, a Engenharia Eletrônica, como uma profissão em constante mutação, responsável por elevado impacto socioeconômico mundial, exige um projeto que direcione ações pedagógicas que contemplem as características supracitadas, mantendo o curso atualizado com as correntes tecnológicas e educacionais contemporâneas, no sentido de oferecer sempre um curso de excelência à sociedade.