\chapter{CONTEXTUALIZAÇÃO}

\section{CONTEXTUALIZAÇÃO NACIONAL, REGIONAL E LOCAL}

%A oferta de cursos de engenharia na UTFPR visa a contribuir com uma preocupação crescente: a carência de bons profissionais da área de engenharia no Brasil. \pdfmarkupcomment{Existem atualmente cerca de 550 mil engenheiros no país, uma razão de seis para cada mil pessoas economicamente ativas, enquanto países como Estados Unidos e Japão têm 25. Quase 50\% dos estudantes de Engenharia no Brasil cursam Engenharia Civil, enquanto em países desenvolvidos há um maior percentual em modalidades ligadas a alta tecnologia. A região Oeste do Paraná possui potencial industrial comprovado, contando com parques industriais estruturados e indústrias nas áreas: alimentos, medicamentos, têxteis, informática e metal mecânica. Além do potencial industrial, a região tem elevada produção agrícola, sendo seus expoentes a suinocultura, avicultura, produção de grãos e leitaria, o que possibilita que inúmeros dispositivos para automação e recursos informatizados possam ser projetados e disponibilizados visando a gestão mais eficiente destas produções.}{É necessário encontrar referências de acordo com o texto norteador}

Os avanços tecnológicos que o Brasil vem testemunhando está alterando significativamente as relações entre as pessoas, o setor privado e o setor público. Neste contexto, os Engenheiros em Eletrônica desempenham um papel de grande importância, catalisando as tecnologias emergentes, levando-as ao encontro das necessidades em diversos setores da economia e da sociedade.


\subsection{O setor do agronegócio}

Um exemplo claro desse importante papel do Engenheiro Eletrônico, ocorre de forma muito forte no setor agroindustrial, onde o estado do Paraná ocupa um lugar de destaque no cenário nacional. A agroindústria do Paraná tem seguido bem de perto as tendências mundiais que apontam para a automatização dos processos de cultivo e colheita na agricultura, consequentemente, produzindo melhorias significativas na produtividade desse setor.

Com grande força agrícola na região oeste do Paraná, o município de Toledo é considerado a ``Capital do Agronegócio do Paraná''. Com seu solo fértil e plano, tornou-se um dos maiores produtores de grãos do estado. Segundo dados de 2021 do IBGE, Toledo possuía 144.601 habitantes \cite{ibge2020}. Seu Índice de Desenvolvimento Humano (IDH) alcançou 0,768 em sua última atualização em 2010. O Produto Interno Bruto (PIB) a preços correntes do município alcançou 6 bilhões de reais em 2019 (cerca de R\$ 44.016,71 per capita). 

\nomenclature[A]{IDH}{Índice de Desenvolvimento Humano}
\nomenclature[A]{IDH}{Índice de Desenvolvimento Humano}

A agricultura de precisão, área do conhecimento que evolui com a participação efetiva de profissionais da área de Eletrônica, tem se utilizado de dispositivos sensores avançados, técnicas de comunicação de dados envolvendo conceitos de Internet das Coisas (IoT) e geoprocessamento, processsamento de imagens, dentre outras tecnologias, para trazer ao mercado soluções tecnológicas que inseridas no setor do agronegócio resulta em maior eficiência e desempenho de processos e produtos. 

Nesse sentido, o curso de Engenharia Eletrônica verificou que egressos evm desenvolvendo tecnologias inovadoras para a setor agrícola, princiapamente na área de processamento de imagens das plantações e monitoramento de grãos.

\subsection{O setor tecnológico}

A UTFPR, com mais de um século de existência, tem se consolidado como uma instituição formadora de recursos humanos na área tecnológica em todo o estado do Paraná. O câmpus da UTFPR na cidade de Toledo foi instalado em 2007, sendo fruto da expansão da UTFPR em direção ao interior por meio do programa de apoio a planos de Reestruturação e Expansão das Universidades Federais (REUNI), tendo também contado com a iniciativa da Prefeitura Municipal de Toledo, da Fundação Educacional de Toledo, além outras autoridades e entidades representativas da região. A carência de profissionais é demonstrada no \autoref{qua:cursoskm}, a qual apresenta os resultados de pesquisas de instituições que oferecem cursos correlatos à Engenharia Eletrônica em um raio de 150 km de Toledo. Foram encontradas apenas três instituições com oferta de cinco cursos, incluindo o presente curso.

\nomenclature[A]{REUNI}{Programa de Apoio a Planos de Reestruturação e Expansão das Universidades Federais}

\begin{quadro}
	\centering\small
	\caption[Cursos de Engenharia eletrônica próximos à Toledo/PR]{Cursos de Engenharia eletrônica próximos à Toledo/PR}
	\begin{tabularx}{\textwidth}{>{\centering\arraybackslash}X >{\centering\arraybackslash}X cc}
		\toprule%
		\rowcolor{white}\bfseries Instituição & \bfseries Nome do Curso & \bfseries Cidade & \bfseries Distância\\
		\midrule
		\csvreader[	head to column names,
					late after line=\csvifoddrow{\\}{\\\rowcolor{gray!10}}, 
					separator=pipe]%
					{Caps/Quadros/cursosProx.csv}{}%
					{\inst & \curso & \cidade & \dist}%
		\bottomrule
	\end{tabularx}
	\label{qua:cursoskm}
\end{quadro}

Neste sentido, o curso de Engenharia Eletrônica vem desempenhando um papel importante ao entregar para a sociedade profissionais especializados, capazes de desenvolverem projetos de engenharia envolvendo tecnologias sempre atualizadas. É notável que diversos egressos estão trabalhando com a confecção de projetos elétricos (residenciais ou industriais), automação residencial e desenvolvimento de sistemas eletrônicos industriais.

\section{CONTEXTUALIZAÇÃO DO CURSO}
\label{sec:const}

\pdfmarkupcomment{Do ponto de vista político, o curso de Engenharia Eletrônica também é fruto da expansão da UTFPR em direção ao interior do Estado do Paraná, através do programa de apoio a planos de Reestruturação e Expansão das Universidades Federais (REUNI)}{Ficou repetido em relação à seçao anterior}. Adicionalmente, o curso contou com a iniciativa da Prefeitura Municipal de Toledo, da Fundação Educacional de Toledo (FUNET) e da UTFPR – Câmpus Medianeira para iniciar as suas atividades. Analisando pelo aspecto econômico, os engenheiros têm papel fundamental para a economia do país, de tal forma que alguns indicadores econômicos se baseiam na atividade de engenharia ou na quantidade de engenheiros formados no país/região. A criação do curso busca atender às necessidades da microrregião de Toledo, mediante formação de profissionais para atuar no setor eletroeletrônico, de automação e de elétrica. \pdfmarkupcomment{A oferta do curso de engenharia eletrônica visa contribuir com uma preocupação crescente: a carência de profissionais da área de engenharia no Brasil. Existem atualmente cerca de 550 mil engenheiros no país, uma razão de seis para cada mil pessoas economicamente ativas, enquanto países como Estados Unidos e Japão têm 25. Quase 50\% dos estudantes de Engenharia no Brasil cursam Engenharia Civil, enquanto em países desenvolvidos há um maior percentual em modalidades ligadas a alta tecnologia, como eletrônica, por exemplo.}{também repette}

\nomenclature[A]{FUNET}{Fundação Educacional de Toledo}

A região Oeste do Paraná possui potencial industrial comprovado, contando com parques industriais estruturados e indústrias nas áreas: alimentos, medicamentos, têxteis, informática e metal mecânica. Além do potencial industrial, a região tem elevada produção agrícola, sendo seus expoentes a suinocultura, avicultura, produção de grãos e leitaria, o que possibilita que inúmeros dispositivos eletrônica para automação e recursos informatizados possam ser projetados e disponibilizados visando a gestão mais eficiente destas produções. Além das evidências regionais, o mercado global e a velocidade dos avanços tecnológicos, principalmente nas áreas de elétrica, eletrônica, automação e computação, têm gerado carência de profissionais qualificados, capazes de atender as demandas e acompanhar tais mudanças. O mercado relacionado à eletrônica é totalmente globalizado, mas com uma forte demanda local, exigindo dos profissionais a apresentação de soluções muitas vezes personalizadas, que demandam menores custos e maior satisfação dos clientes. Adicionalmente, percebe-se o avanço do emprego de dispositivos eletrônicos nas mais diversas áreas, incluindo o agronegócio, processamento de alimentos, indústria química e os outros ramos da engenharia. Desta forma, um profissional que combine um sólido embasamento em sua área de atuação, com forte perfil inovador, e liderança estará melhor adaptado e mais propenso a atender as necessidades locais.

Considerando o cenário futuro, a direção do Câmpus Toledo da UTFPR, com apoio da Reitoria e da Pró-reitoria de Relações Empresariais e Comunitárias (PROREC) firmaram um convênio com o Biopark, um empreendimento privado, sediado em Toledo e que visa se transformar em um parque tecnológico referência em biociências e biotecnologia. A UTFPR recebeu a doação de um terreno com uma área de 37.375 m$^2$ para a instalação de um complexo da Universidade Tecnológica Federal do Paraná (UTFPR). O estabelecimento desse parque tecnológico deve fortalecer ainda mais a região na área de tecnologia, aumentando a demanda por engenheiros. 

\nomenclature[A]{PROREC}{Pró-reitoria de Relações Empresariais e Comunitárias}

A evolução Histórica do curso de Engenharia Eletrônica pode ser resumida como:

\begin{enumerate}
	\item 	Ingresso da primeira turma em 2009 com a denominação de curso de Engenharia Industrial Elétrica com Ênfase em Automação;
	\item 	Mudança do nome do curso para Engenharia Eletrônica no início de 2010;
	\item	Em 2012, ocorreu o reconhecimento do curso pelo MEC, que atribuiu conceito 4 ao curso;
	\item	Em 2013, ocorreu a formatura da primeira turma;
	\item	Em 2017, o curso foi avaliado no ENADE, obtendo conceito 5;
	\item	Em 2019, o curso completou 10 anos de existência e foi submetido a uma nova avaliação do ENADE, obtendo conceito 4.
\end{enumerate}

Com base neste contexto, a UTFPR estruturou seu curso de Engenharia Eletrônica, que oferecerá uma formação ampla e diversificada, dentro da grande área da Elétrica, que inclui as áreas básicas Matemática, Física, Química, Informática e Humanas, que, visam proporcionar melhores condições para as práticas. Também engloba áreas mais aplicadas, as de cunho profissionalizante, tais como Eletrônica Analógica e Digital, Automação e Controle, Processamento Digital de Sinais e Sistemas Embarcados/Microcontrolados.
 
Dessa forma, o egresso do curso de Engenharia Eletrônica pode atuar em diversas áreas, abrangendo indústrias de materiais, dispositivos e instrumentos elétricos, eletrônicos e de informática, escritórios de engenharia, empresas de geração e distribuição de energia, empresas de consultoria e assessoramento, empresas de software, serviços públicos e instituições de ensino e pesquisa, produção industrial, desenvolvimento de software, gestão de pessoas e de processos, desenvolvimento de hardware e software para os processos de automação. Integrado a esse contexto, o curso de Engenharia Eletrônica tem papel fundamental na região, contribuindo para o seu desenvolvimento.

\section{QUADRO DE DADOS GERAIS DO CURSO}

O \autoref{qua:geral} representa os dados gerais do curso.

\begin{quadro}
	\centering\small
	\caption[Dados gerais do curso]{Dados gerais do curso}	
	\label{qua:geral}
	\begin{tabularx}{0.8\textwidth}{|>{\raggedleft\arraybackslash}X || >{\raggedright\arraybackslash}X|}
		\hline
		\csvreader[	head to column names,
					separator=pipe,
					late after line=\csvifoddrow{\\}{\\\rowcolor{gray!10}}, 
					table head=\hline, 
					table foot=\hline]%
					{Caps/Quadros/quadroGeral.csv}{}{%
						\tipo & \dado 
					}
		\hline
	\end{tabularx}
	
\end{quadro}

\section{FORMA DE INGRESSO E VAGAS}

O acesso aos cursos superiores da UTFPR desde o ano de 2009 ocorre de acordo com o Sistema de Seleção Unificado (SISU) que utiliza a nota do Exame Nacional do Ensino Médio (ENEM), conforme a Deliberação n\textordmasculine 04/2009 do Conselho Universitário da UTFPR \cite{sisuutfpr}.

\nomenclature[A]{SISU}{Sistema de Seleção Unificado}
\nomenclature[A]{ENEM}{Exame Nacional do Ensino Médio}

As entradas definidas no parágrafo anterior compreendem 44 vagas semestrais, sendo que os alunos ingressantes iniciam o curso no período vespertino, havendo alternância dos próximos períodos para matutino (períodos pares) e vespertino (períodos ímpares). Tal oferta de vagas é definida em decorrência dos parâmetros do MEC quanto à liberação de vagas docentes no ato da autorização dos cursos e também do planejamento da própria UTFPR em termos das dimensões das salas de aulas teóricas e práticas.

Também são admitidos alunos por meio de editais de processos seletivos para vagas remanescentes ou transferência a partir do segundo semestre, obedecendo às normas aprovadas pelo Conselho de Graduação e Educação Profissional (COGEP) da UTFPR.

\section{OBJETIVOS DO CURSO}
\label{sec:obj}

\pdfmarkupcomment{Em função do planejamento estratégico institucional e das ações definidas pelo planejamento do curso foram definidos os objetivos descritos abaixo:}{RECOMENDA-SE REDUZIR ESSE TEXTO}

\begin{enumerate}
	\item	Formar um profissional generalista, que atua na área de materiais eletroeletrônicos; sistemas de medição e de controle eletroeletrônico; desenvolvimento de sistemas, produtos e equipamentos eletrônicos, sistemas embarcados, conversores de energia e instalações elétricas.
	\item	Formar um profissional que estuda, projeta e especifica materiais, componentes, dispositivos e equipamentos eletroeletrônicos, eletromecânicos, magnéticos, ópticos, de instrumentação, sensores e atuadores de transmissão e recepção de dados, de áudio/vídeo, de segurança patrimonial e de eletrônica embarcada.
	\item	Capacitar o graduado a planejar, projetar, instalar, operar e manter sistemas e instalações eletrônicas, instalações elétricas, equipamentos, sistemas de medição e instrumentação eletroeletrônica, de acionamentos de máquinas elétricas, de controle eletrônico e de automação e de sistemas eletrônicos embarcados.
	\item	Fornecer embasamento teórico para coordenar e supervisionar equipes de trabalho, realizar estudos de viabilidade técnico-econômica, executar e fiscalizar obras e serviços técnicos; e efetuar vistorias, perícias e avaliações, emitindo laudos e pareceres.
	\item	Imbuir no profissional a ética, a segurança, a legislação e os impactos ambientais.
	\item	Fornecer um embasamento sólido que permita ao aluno dar prosseguimento a seus estudos, expandindo sua área de atuação, atuando em áreas multidisciplinares, ou buscando pós-graduação.
	\item	Atender a legislação profissional, habilitando o graduado a atuar em um amplo espectro da Engenharia Elétrica e Eletrônica, com atribuições condizentes com as resoluções relativas a atribuições profissionais do Conselho Federal de Engenharia e Agronomia (CONFEA) e do Conselho Regional de Engenharia e Agronomia (CREA).
	\item	Estabelecer-se como um curso flexível permitindo ao aluno participar de programas de mobilidade acadêmica, de intercâmbios e de programas de dupla diplomação.
	\item	Permitir a celebração de convênios de dupla diplomação com universidades estrangeiras.
	\item	Permitir ao egresso do curso a atualização constante através de disciplinas optativas nas áreas de aprofundamento, com a possibilidade de serem cursadas em outros campi da UTFPR, facultando-lhe agregar novas competências e atribuições profissionais junto ao sistema CONFEA/CREA.
\end{enumerate}

Pretende-se que o Curso venha a se distinguir pela acentuada integração com empresas, pela busca de integração com cursos de Pós-Graduação, pela significativa visão sistêmica e integração entre software e hardware, pela integração das disciplinas de Trabalho de Conclusão de Curso (TCC), pela educação continuada, pela possibilidade de convalidação de créditos cursados em Universidades estrangeiras, pelo incentivo ao empreendedorismo, pela diversidade das Áreas de Conhecimento, bem como pela elevada carga horária em laboratórios.

\section{PERFIL DO EGRESSO}
\label{sec:perf}

Além das características estabelecidas no Art. 3\textordmasculine{} das DCNs, o egresso do curso superior de Bacharelado em Engenharia Eletrônica é um profissional versátil capaz de propor soluções em sistemas eletroeletrônicos em contexto local e global, considerando legislação, normas técnicas, preceitos ético-políticos, sustentabilidade, inovações, e bem estar social. Tal profissional se caracteriza por conceber sistemas analógicos, de potência, de processamento digital, de controle automação e eletrotécnica. Poderá atuar em empresas públicas ou privadas de base tecnológica, no desenvolvimento de hardware e software e a sua integração com outros sistemas, bem como na capacitação de equipes de profissionais da área tecnológica. Sendo capaz de:


\renewcommand{\labelenumi}{\roman{enumi}}
\begin{enumerate}
	\item Desenvolver sistemas eletroeletrônicos eficazes, gerenciando os recursos tecnológicos, de forma sustentável; 
	\item Gerenciar o desenvolvimento de projetos, segundo normas e critérios técnicos de segurança e de desempenho, com senso crítico e atitude colaborativa. 
	\item Capacitar equipes de profissionais da área tecnológica, com comunicação qualificada.
\end{enumerate} 

As DCNs \cite{dcneng} estabelecem competências gerais em seu Artigo 4\textordmasculine, indicando também no parágrafo único deste Artigo o estabelecimento de competências específicas, de acordo com a habilitação e a ênfase do curso. Dessa forma, as Competências específicas do curso de Engenharia Eletrônica redigidas pelo NDE são:

\begin{itemize}
	\item \textbf{Eletrônica}: conceber e/ou intervir em sistemas eletrônicos com autonomia, integrando circuitos analógicos, computação embarcada, controle de sistemas e 	processamento digital de informações e considerando uma documentação clara e concisa;
	\item \textbf{Científica}: produzir investigação científica integrando modelos de fenômenos naturais, conhecimentos técnico-científicos, escrita e metodologia científica com honestidade intelectual e senso crítico;
	\item \textbf{Empreendedora}: Analisar ou propor negócios, com responsabilidade compartilhada e atitudes empreendedora e cooperativa, por meio da articulação de informações técnicas e conceituais e avaliação do micro e macroambiente;
	\item \textbf{Eletrotécnica}: Conceber e/ou intervir em sistemas elétricos com autonomia, integrando instalações elétricas , máquinas elétricas e sistemas de potência, considerando uma documentação clara e concisa e segurança elétrica.
\end{itemize}

O desenvolvimento dessas competências, no âmbito do curso de Engenharia Eletrônica, é discutido na \autoref{sec:comp}.
