\quadroUnitCurricularLegenda{yellow} % Usar o comando definido


\quadroUnitCurricular{6.1}{1}

\quadroUnitCurricular{1.1}{0.7}
%\dadoCHT{1.1}

\newcounter{a}  % Valor do período
\newcounter{b}  % Contador temporário para somar teórica com prática
\setcounter{a}{1}
\setcounter{b}{1}
% \quadroUnitCurricular{\arabic{a}.\arabic{b}}{1}
\defineCorFromCSV{\arabic{a}.\arabic{b}}
\corQuadro

\setcounter{a}{6}
\setcounter{b}{1}
% \quadroUnitCurricular{\arabic{a}.\arabic{b}}{1}
\defineCorFromCSV{\arabic{a}.\arabic{b}}
\corQuadro

\setcounter{a}{1}
\setcounter{b}{5}
% \quadroUnitCurricular{\arabic{a}.\arabic{b}}{1}
\defineCorFromCSV{\arabic{a}.\arabic{b}}
\corQuadro

\setcounter{a}{8}
\setcounter{b}{5}
% \quadroUnitCurricular{\arabic{a}.\arabic{b}}{1}
\defineCorFromCSV{\arabic{a}.\arabic{b}}
\corQuadro
% \maxdisciplinas

\quadroTituloPeriodo{1}{1}



\dadoCHT{\value{a}.\value{b}}

\quadroResumoPeriodo{1}{1}

\quadroContabilizacao
% \newpage


% \begin{landscape}
% % Inserir um novo float em uma página separada em formato paisagem

    
    % \clearpage % Certifique-se de começar uma nova página
    \KOMAoptions{paper=landscape,paper=A3}
    \recalctypearea
    \newgeometry{margin=1cm, bottom=0cm} 
    % \begin{landscape}
    \thispagestyle{plain}
    % \pagenumbering{gobble}
    \begin{figure}[p]
        \centering
        \distribuiQuadrosUC
        \caption{Quadro de disciplinas}
    \end{figure}
    
        % Seu conteúdo do quadro vai aqui
    % \imprimeQuadroUC

  
    
    % \end{landscape}
    \restoregeometry % Restaura a geometria do documento para o padrão após o quadro
    % \pagenumbering{arabic}
    \KOMAoptions{paper=portrait,paper=A4}
    \recalctypearea

% \end{landscape}